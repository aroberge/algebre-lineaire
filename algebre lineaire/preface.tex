\chapter{Préface}

\begin{quote}\small \textit{
On ne termine jamais un livre \ldots on l'abandonne à un éditeur.}
\end{quote}
\sourceatright{\small auteur inconnu}

\begin{quote}\small \textit{
I prided myself in reading quickly. I was really amazed by my first encounters with serious mathematics textbooks. I was very interested and impressed by the quality of the reasoning, but it was quite hard to stay alert and focused. After a few experiences of reading a few pages only to discover that I really had no idea what I'd just read, I learned to drink lots of coffee, slow way down, and accept that I needed to read these books at 1/10th or 1/50th standard reading speed, pay attention to every single word and backtrack to look up all the obscure numbers of equations and theorems in order to follow the arguments.}
\end{quote}
\sourceatright{\small William Thurston, mathématicien célèbre}

L'algèbre linéaire est une branche des mathématiques qui s'intéresse aux espaces linéaires, plus généralement nommés espaces vectoriels, incluant l'étude des vecteurs et des équations linéaires ainsi que les transformations linéaires.  
Les techniques développées dans l'étude de l'algèbre linéaire 
sont utilisées dans tous les domaines scientifiques.

Dans ce manuel, je présente une introduction à l'algèbre linéaire. 
Plutôt que d'avoir un texte purement axé sur une approche formelle, 
c'est-à-dire basé uniquement sur une suite de théorème et démonstration,
j'ai incorporé une multitude d'exemples.  Le lecteur\footnote{Pour simplifier l'écriture, je vais généralement utiliser le masculin pluriel comme terme inclusif pour désigner aussi bien
les hommes que les femmes.} doit cependant se rappeler qu'un exemple n'est pas une preuve, et qu'on ne peut pas toujours généraliser
à partir d'un exemple.  Cependant, je suis d'avis que les exemples sont utiles comme guide pour mieux comprendre
l'application des théorèmes.

\section{Autres ressources}

On retrouve sur Internet plusieurs sites, tel que Wikipédia, qui offrent des explications et des dérivations mathématiques.  
On retrouve également des vidéos explicatives, tel que sur le site \url{www.KhanAcademy.com} ou encore à partir d'universités,
tel que l'université Massachusetts Institute of Technology et son MITOpenCourseware, pour n'en nommer qu'une parmi tant d'autres.
Plusieurs professeurs mettent également des copies de leurs notes de cours avec un accès complètement libre sur Internet.
J'encourage le lecteur à explorer ces différentes ressources pour approfondir sa connaissance du sujet.
Puisqu'il existe beaucoup plus de ressources en anglais qu'en français, je fais mention à l'occasion des termes utilisés en
anglais, surtout lorsque la traduction est loin d'être évidente, comme par exemple \textit{cross product} (produit vectoriel),
ou \textit{eigenvalue} (valeur propre) de façon à permettre aux intéressés de comprendre plus facilement le 
matériel disponible en anglais.

\section{Remerciements}

J'aimerais remercier mes enfants, Julien et Evelyne, ainsi que Alain Gamache et Dany Sheehy pour m'avoir communiqué leurs impressions, commentaires et suggestions après avoir lu une ébauche des premiers chapitres de ce manuel.  

J'aimerais également remercier Monsieur Marcel B. Finan de l'Arkansas Tech University qui m'a donné la
permission d'adapter ses notes de cours sur le même sujet; ces notes ont surtout été utilisées dans la préparation
des premiers chapitre.  Je remercie également Monsieur Joseph Khoury de l'Université d'Ottawa pour m'avoir
donné la permission d'utiliser et d'adapter les exemples d'applications de l'algèbre linéaire qui 
se trouve sur son site Internet \url{http://aix1.uottawa.ca/~jkhoury/linearnewf.htm}.

Finalement, des étudiants ont noté des coquilles et ont offerts des suggestions qui m'ont permis d'améliorer
ce manuel.  En ordre alphabétique, j'aimerais donc remercier Colin Bonnar, Valérie Carroll, Natalia Ensor, Marie-Josée Guyon,
Mathieu Manuel, Andrée-Anne Rousselle et Lianne Saulnier  pour leurs diverses contributions.

\section{Note sur l'utilisation des couleurs}

Dans ces notes, j'utilise parfois des couleurs comme  le \textcolor{blue}{bleu} et le \textcolor{red}{rouge} pour accentuer certains termes.  Si vous êtes daltonien et que vous ne distinguez pas certaines des couleurs que j'utilise, contactez-moi pour que je puisse préparer une autre version avec des couleurs plus appropriées pour vous.
