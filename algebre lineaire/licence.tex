
\chapter{Licence \ccLogo \ccAttribution \ccNonCommercial \ccShareAlike}

\setlength{\parindent}{0pt}
{\large
Ce manuel a été conçu par André Roberge. \bigskip

Ce  manuel est publié sous la licence {\bfseries Creative Commons \ccLogo} suivante:

Attribution \ccAttribution; pas d'utilisation commerciale \ccNonCommercial; partage à l'identique \ccShareAlike; 3.0 non transposé (CC BY-NC-SA 3.0)
}

{\small
Ce qui suit est le résumé\footnote{adapté de http://creativecommons.org/licenses/by-nc-sa/3.0/deed.fr} explicatif "lisible par les humains" du Code Juridique (la version intégrale de la licence):
\smallskip

Vous êtes libre de :\smallskip

{\bfseries partager} — reproduire, distribuer et communiquer l'oeuvre\\
{\bfseries remixer} — adapter l'oeuvre\smallskip

selon les conditions suivantes :\smallskip

{\Huge \ccAttribution} {\bfseries Attribution} — Vous devez attribuer l'oeuvre de la manière indiquée par l'auteur de l'oeuvre ou le titulaire des droits (mais pas d'une manière qui suggérerait qu'ils vous soutiennent ou approuvent votre utilisation de l'oeuvre).\\
{\Huge \ccNonCommercial} {\bfseries Pas d’utilisation commerciale} — Vous n'avez pas le droit d'utiliser cette oeuvre à des fins commerciales.\\
{\Huge \ccShareAlike} {\bfseries Partage à l'identique} — Si vous modifiez, transformez ou adaptez cette oeuvre, vous n'avez le droit de distribuer votre création que sous une licence identique ou similaire à celle-ci.\smallskip

Comprenant bien que :
{\bfseries Renoncement} — N'importe laquelle des conditions ci-dessus peut être modifiées si vous avez l'autorisation du titulaire de droits.\\
{\bfseries Domaine public} — Là où l'oeuvre ou un quelconque de ses éléments est dans le domaine public selon le droit applicable, ce statut n'est en aucune façon affecté par la licence.\\
{\bfseries Autres droits} — Les droits suivants ne sont en aucune manière affectés par la licence :
\begin{itemize}
\item Vos prérogatives issues des exceptions et limitations aux droits exclusifs ou usage juste ("fair use");
\item Les droits moraux de l'auteur;
\item Droits qu'autrui peut avoir soit sur l'oeuvre elle-même soit sur la façon dont elle est utilisée, comme le droit à l'image ou les droits à la vie privée.
\end{itemize}
{\bfseries Remarque} — A chaque réutilisation ou distribution de cette oeuvre, vous devez faire apparaître {\bfseries clairement} au public la licence selon laquelle elle est mise à disposition.
}
