\documentclass[11pt]{article}
\textwidth=6in
\textheight=9.5in
\topmargin=-1in
\headheight=0.5in

\hoffset  -.85in

\usepackage[utf8]{inputenc}			% enable input of accented letters
\usepackage[T1]{fontenc}				% related to above
\usepackage[french]{babel}			% French language support
\usepackage{translator}				% same
	\uselanguage{French}
	\languagepath{French}

\usepackage{url}
\usepackage{fancyhdr}
\pagestyle{fancy}

\lhead{}
\rhead{}
\chead{\bfseries MATH 1253 Introduction à l'algèbre linéaire -- automne 2012}
\lfoot{}
\cfoot{}
\rfoot{\thepage}
\renewcommand{\headrulewidth}{0.4pt}

\begin{document}

\noindent\textbf{Professeur: } André Roberge \\
\textbf{Adresse courriel: } Andre.Roberge@usainteanne.ca\\
\textbf{Site du cours: } \url{moodle.usainteanne.ca} \\
\textbf{Horaire des cours:} Lundis et jeudis, de 10h à 11h30 {\tiny [voir ci-dessous]} \\
\textbf{Manuel de cours:} \textit{Introduction à l'algèbre linéaire} en format pdf.

\paragraph*{Description} 

{\small
\textit{MATH 1253: Introduction à l'algèbre linéaire}
Revue des opérations de base sur les matrices (addition, soustraction et multiplication) et sur la résolution de simples systèmes d'équations linéaires.  Méthode d'élimination de Gauss-Jordan.  Matrice inverse. Déterminants. Espaces et sous-espaces vectoriels, base et dimension.  Espaces euclidiens. Vecteurs propres et valeurs propres.  Transformations linéaires et applications diverses.
}

\paragraph*{Horaire des cours} Selon la grille horaire préparée, les cours sont donnés en principe 
par vidéo-conférence les lundis et jeudis, de 10h à 11h30.  Cependant, il est possible que, sauf pour
les deux tests, vous n'ayez pas à vous présenter en classe aux heures prévues.

En effet, je propose d'utiliser un modèle de \textit{classe inversée}, où les heures de classe prévues à
l'horaire sont utilisées comme heures de consultation (individuelle ou en groupe) sur rendez-vous\footnote{Notez que si personne ne prend rendez-vous à l'avance, je ne serai pas disponible par vidéoconférence.
}, alors
que les classes traditionnelles sont remplacées par de courtes vidéos disponibles,
qui incluent des mini-quiz qui ne comptent pas envers la note finale.   En raison de cette approche,
vous serez libre d'agencer votre horaire comme il vous plaira.  De plus, ceci vous permettra de revoir
plusieurs fois les points jugés plus difficiles à comprendre, puisque les vidéos seront disponibles sur demande.
\textbf{Veuillez noter que cette méthode est utilisée sur une base expérimentale et pourrait changer.} En effet, si, \textit{au plus tôt} après que les
trois premières semaines de cours soient complétées, je détermine que le format ne produit pas les résultats escomptés, ou si un nombre
suffisant d'étudiants indiquent vouloir avoir des classes régulières par vidéoconférence plutôt que des vidéos
pré-enregistrées, nous ferons le changement souhaité pour revenir à un format plus traditionnel.  

\paragraph*{Utilisation de moodle}  Tout le matériel pour le cours (manuel, vidéos, etc.) est disponible via la plate-forme moodle. 
Je vous encourage à utiliser le forum de discussion pour soit poser vos questions, répondre aux questions de vos collègues
(ce qui est une bonne façon d'approfondir son apprentissage de la matière), ou faire des commentaires sur le manuel ou les vidéos.
Je vous encourage \textbf{fortement} à vérifier le site du cours à chaque jour pour voir s'il y a des nouveautés.

\paragraph*{Devoirs} Il y aura des devoirs à remettre à chaque semaine, le mercredi avant midi. 
Ces devoirs doivent être remis à la réception, à l'attention de Jason Saulnier, facilitateur.
 Tout devoir remis en retard recevra une note de zéro.   
 Veuillez noter que la quantité de travail requis ne sera pas nécessairement uniforme d'une semaine à une autre.  
 Puisque vous aurez accès au matériel requis bien avant d'avoir à remettre un devoir,
  je vous encourage fortement à prendre de l'avance sur l'horaire prévu pour la remise des devoirs.  
  La liste des problèmes à faire pour chaque devoir sera mise à jour sous peu.

Vous devriez faire vos devoirs individuellement.  Cela dit, je vous encourage fortement à discuter entre vous des exemples du manuel ou
du matériel présenté sous forme de vidéos.   Si vous avez des questions, posez-les de préférence dans le forum de discussion pour que
les réponses fournies puissent être utilisées par tout le monde.

\paragraph*{Détermination de la note finale} En plus des devoirs, il y aura deux tests et un examen final dans ce cours.

En premier, le calcul de la note se fait avec une pondération relative de 15 pour le premier test,
de 20 pour le deuxième test, et de 50 pour l'examen final -- donc pour un total de 85 qui ne tient
pas compte des devoirs.  La note est
convertie en pourcentage; \textbf{si ce résultat est inférieur à 50\%, l'étudiant reçoit la note E pour le cours.}

\marginpar{
\begin{tabular}{|l|l|}
\hline \rule{0pt}{0pt} A+ & 94 -- 100 \\   % adjust up to 4 marks lower i.e. 90 -- 100
\hline \rule{0pt}{0pt} A & 87 -- 94\\     % adjust up to 3 marks lower i.e. 85 -- ...
\hline \rule{0pt}{0pt} A- & 82 -- 87 \\    % from here on, adjust up to 2 marks lower i.e. 80 - ...
\hline \rule{0pt}{0pt} B+ & 78 -- 82 \\    % 76 -
\hline \rule{0pt}{0pt} B & 74 -- 78 \\     % 72-
\hline \rule{0pt}{0pt} B- & 71 -- 74\\    % 70-
\hline \rule{0pt}{0pt} C+ & 67 -- 71 \\   % do not adjust the threshold from here on 
\hline \rule{0pt}{0pt} C & 63 -- 67 \\ 
\hline \rule{0pt}{0pt} C- & 59 -- 63\\ 
\hline \rule{0pt}{0pt} D & 50 -- 59 \\ 
\hline \rule{0pt}{0pt} E &  $<50$\\ 
\hline 
\end{tabular} 
}

Si le résultat précédent est égal ou supérieur à 50\%, le calcul de la note est fait d'une autre
façon en tenant compte des résultats des devoirs auxquels est attribué une pondération relative de 15,
en ayant toujours une pondération relative de 15 pour le premier test, 20 pour le deuxième et 50 pour le troisième.
Pour chaque étudiant, sur une base individuelle, la méthode qui mène à la note en pourcentage la plus élevée est
celle qui est utilisée. Par la suite, la note en
pourcentage est convertie en note alphabétique selon le tableau qui apparait dans la marge.
\footnote{Il peut arriver que des notes soient majorées vers le haut si je juge que faire autrement serait injuste. Par exemple, supposons que
les deux notes les plus hautes obtenues dans le cours sont 94 \% et 92\% et que les
autres étudiants ont obtenus des notes beaucoup plus basse; dans ce cas, les deux étudiants
qui auraient obtenu les plus hautes notes auraient, à toutes fins pratiques, démontré une
maîtrise semblable de la matière et l'étudiant(e) qui
aurait obtenu une note de 92\% verrait sa note convertie à A+.}

{\small
\paragraph*{Dates importantes:}
\begin{center}
\begin{tabular}{|l|l|}
\hline \rule{0pt}{0pt}  jeudi 6 sept. & première classe  \\ 
\hline \rule{0pt}{0pt}  mercredi 12 sept. &  Devoir 1: exercices 1.1--1.6, 1.8, 1.10, 1.11\\ 
\hline \rule{0pt}{0pt}  mercredi 19 sept. &  Devoir 2, exercices 2.2, 2.5, 2.9--2.11, 2.14--2.17, 2.19\\ 
\hline \rule{0pt}{0pt}  mercredi 26 sept. &  Devoir 3, exercices 3.17, 3.19--3.21, 3.24--3.26, 3.29, 3.31, 3.35, 3.36\\ 
\hline \rule{0pt}{0pt}  mercredi 3 oct. &  Devoir 4, chapitre 4\\
\hline \rule{0pt}{0pt}  mercredi 10 oct. &  Devoir 5, chapitre 5\\
\hline \rule{0pt}{0pt}  jeudi 11 oct. &  Test, chapitres 1 -- 4 \\ 
\hline \rule{0pt}{0pt}  mercredi 17 oct. &  Devoir 6, chapitre 6\\
\hline \rule{0pt}{0pt} 22 au 26 oct.&  Semaine d'étude et de consultation\\ 
\hline \rule{0pt}{0pt}  mercredi 31 oct. &  Devoir 7, chapitre 7\\ 
\hline \rule{0pt}{0pt}  mercredi 7 nov. &  Devoir 8, chapitre 8\\ 
\hline \rule{0pt}{0pt}  mercredi 14 nov. &  Devoir 9, chapitre 9\\ 
\hline \rule{0pt}{0pt}  jeudi 15 nov. &  chapitres 1 -- 8\\ 
\hline \rule{0pt}{0pt}  mercredi 21 nov. &  Devoir 10, chapitre 10\\ 
\hline \rule{0pt}{0pt}  mercredi 28 nov. &  Devoir 11, simulation d'un examen final  \\ 
\hline \rule{0pt}{0pt} 10 au 16 déc. &  Examen final, date exacte à déterminer\\ 
\hline 
\end{tabular} 
\end{center}
}

\paragraph*{Ressources supplémentaires: } Le manuel gratuit est, en principe, la seule ressource requise pour apprendre la matière.
Les leçons vidéos sont une autre façon de présenter le même matériel. Vous ne devriez pas avoir besoin d'utiliser d'autre matériel.\footnote{
Pour ceux qui tiennent absolument à dépenser de l'argent pour acheter un  livre, je vous suggère le livre d'algèbre linéaire de  David Lay
que vous pourrez trouver dans une librairie en ligne (Chapters ou Amazon) au coût approximatif de 110\$.}

Si vous pensez autrement, je vous encourage à faire une recherche pour \textit{algèbre linéaire} dans la base de données 
de la bibliothèque Louis R. Comeau.  Vous y verrez une quinzaine de livres sur le sujet qui sont essentiellement au même niveau
que le cours.  Il existe également toute une gamme de ressources disponibles sur Internet; en particulier, le site \url{KhanAcademy.org}
contient toute une collection de vidéos sur plusieurs sujets dont l'algèbre linéaire.  Ce site a servi en partie d'inspiration
pour la production des vidéos du cours MATH 1253.



\end{document}