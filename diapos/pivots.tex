%\documentclass[french, handout]{beamer}
\documentclass[french]{beamer}

\usetheme{Boadilla}
\usefonttheme{structuresmallcapsserif}

% % %  default French language settings
\usepackage[utf8]{inputenc}			% enable input of accented letters
\usepackage[T1]{fontenc}				% related to above
\usepackage[french]{babel}			% French language support

\usepackage{diapos}

%\usepackage{tikz}					% for drawing figures
%\usetikzlibrary{positioning}
%\tikzset{>=stealth}
%\newcommand{\tikzmark}[3][]{\tikz[overlay,remember picture,baseline] \node [anchor=base,#1](#2) {#3};}

	
\title{Systèmes d'équations linéaires}
\author{Math 1253} % shows on bottom of each page by default; students do not care about
    % my name, nor the date in which this was written however, if a revision needs to 
    % be done, having a version number could be useful
\date{version 1}   % shows on bottom of page

\begin{document}
	\frame{\titlepage}
	
	\begin{frame}{Système d'équations linéaires}
	\begin{itemize}
	\item Questions?
	\item Forme échelonnée générale
	\begin{itemize}
		\item coefficient principal ou pivot
		\item inconnue principale, ou variable liée, ou variable dépendante
		\item inconnue non principale, ou variable libre, ou variable indépendante
	\end{itemize}
	\item Équations paramétriques
	\begin{itemize}
	\item équation d'un cercle
	\item équation d'une droite
	\item solution générale d'un système d'équations linéaires
	\end{itemize}
	\item Équilibrage des réactions chimiques
	\end{itemize}
\end{frame}	

\begin{frame}{Forme échelonnée}
Une matrice est en forme échelonnée si
le premier coefficient non nul d'une ligne donné est toujours à 	la droite du coefficient non nul de la ligne précédente.

\[
\begin{pmatrix}
p_1 & *  & * & * & * & * & * & * & * \\
{\color{red}{0}} & {\color{red}{0}} & p_2 & * & * & * & * & * & * \\
{\color{red}{0}} & {\color{red}{0}} & {\color{red}{0}} & p_3 & * & * & * & * & * \\ 
{\color{red}{0}} & {\color{red}{0}} & {\color{red}{0}} & {\color{red}{0}} & {\color{red}{0}} & {\color{red}{0}} & p_4 & * & * \\ 
{\color{red}{0}} & {\color{red}{0}} & {\color{red}{0}} & {\color{red}{0}} & {\color{red}{0}} & {\color{red}{0}} & {\color{red}{0}} & {\color{red}{0}} & p_5 \\ 
{\color{red}{0}} & {\color{red}{0}} & {\color{red}{0}} & {\color{red}{0}} & {\color{red}{0}} & {\color{red}{0}} & {\color{red}{0}} & {\color{red}{0}} & {\color{red}{0}} 
\end{pmatrix}
\]
Ces premiers coefficients non-nuls d'une ligne donné s'appellent les \textbf{pivots}, ou 
coefficients principaux.
\end{frame}


\begin{frame}{Forme échelonnée réduite}
Une matrice est en \textbf{forme échelonnée réduite} si elle est en forme échelonnée,
que tous ses pivots valent 1 et que les autres coefficients dans les colonnes des
pivots sont nuls, comme dans la matrice suivante:
\[
\begin{pmatrix}
1 & * & {\color{blue}{0}} & {\color{blue}{0}} & * & * & {\color{blue}{0}} & * & {\color{blue}{0}}  \\
{\color{red}{0}} & {\color{red}{0}} & 1 & {\color{blue}{0}} & * & * & {\color{blue}{0}} & * & {\color{blue}{0}}  \\
{\color{red}{0}} & {\color{red}{0}} & {\color{red}{0}} & 1 & * & * & {\color{blue}{0}} & * & {\color{blue}{0}}  \\ 
{\color{red}{0}} & {\color{red}{0}} & {\color{red}{0}} & {\color{red}{0}} & {\color{red}{0}} & {\color{red}{0}} & 1 & * & {\color{blue}{0}}  \\ 
{\color{red}{0}} & {\color{red}{0}} & {\color{red}{0}} & {\color{red}{0}} & {\color{red}{0}} & {\color{red}{0}} & {\color{red}{0}} & {\color{red}{0}} & 1  \\ 
{\color{red}{0}} & {\color{red}{0}} & {\color{red}{0}} & {\color{red}{0}} & {\color{red}{0}} & {\color{red}{0}} & {\color{red}{0}} & {\color{red}{0}} & {\color{red}{0}}  
\end{pmatrix}
\]

\end{frame}
	
\begin{frame}{Unicité de la forme échelonnée réduite}
\begin{theorem}
	Toute matrice est équivalente selon les lignes à une et une seule matrice échelonnée réduite.
\end{theorem}

\end{frame}


\begin{frame}{Variable dépendantes}
\begin{definition}
Dans un système d'équation linéaires en forme échelonnée réduite, les \textbf{variables dépendantes}
(ou inconnues principales, ou variable liées)
correspondent aux pivots; les \textbf{variables indépendantes} également appelées
variables libres, ou inconnues non principales, sont les autres variables.  
\end{definition}

\end{frame}

\begin{frame}{Exemple}
	Déterminez quelles sont les variables dépendantes et les variables indépendantes\footnote{La plupart du temps, toutes les variables sont équivalentes: ce n'est que lorsqu'on les ordonne dans des colonnes différences que certaines sont identifiées comme étant \textit{dépendantes} et d'autres comme étant \textit{indépendantes}.} du système suivant:
	\[
	\left\{
	\begin{matrix}[rrrrrrrrrrrrr]
	x_1 &+& 3x_2 &-& 2x_3 &\ppp&\ppp   &+& 2x_5  &\ppp&\ppp  &=& 0 \\
	2x_1 &+& 6x_2 &-& 5x_3 &-& 2x_4 &+& 4x_5 &-& 3x_6 &=& -1 \\
	&&  && 5x_3 &+& 10x_4 && &+& 15x_6 &=& 5 \\
	2x_1 &+& 6x_2 &&  &+& 8x_4 &+& 4x_5 &+& 18x_6 &=& 6
	\end{matrix}
	\right.
	\]
\end{frame}

\begin{frame}{Exemple}
	La matrice augmentée de ce système est
	\[
	\begin{bmatrix}[rrrrrr|r]
	1 & 3 & -2 & 0 & 2 & 0 & 0 \\
	2 & 6 & -5 & -2 & 4 & -3 & -1\\
	0 & 0 & 5 & 10 & 0 & 15 & 5\\
	2 & 6 & 0 & 8 & 4 & 18 & 6
	\end{bmatrix}
	\]
	Nous pouvons utiliser la procédure d'élimination de Gauss de la façon suivante:
	\[
	\begin{matrix}[rcl]
		\begin{matrix}
		L_2 - 2L_1 \rightarrow L_2 \\[5pt]
		\frac{1}{5}L_3 \rightarrow L_3 \\[5pt]
		L_4 - 2 L_1 \rightarrow L_4
		\end{matrix}
		&\Longrightarrow&
		\begin{bmatrix}[rrrrrr|r]
		1 & 3 & -2 & 0 & 2 & 0 & 0 \\
		0 & 0 & -1 & -2 & 0 & -3 & -1\\
		0 & 0 & 1 & 2 & 0 & 3 & 1\\
		0 & 0 & 4 & 8 & 0 & 18 & 6
		\end{bmatrix}
      \\[25pt]
		\begin{matrix}
		L_3  + L_2 \rightarrow L_3 \\[5pt]
		L_4 +4L_2 \rightarrow L_4
		\end{matrix}
		&\Longrightarrow&
		\begin{bmatrix}[rrrrrr|r]
		1 & 3 & -2 & 0 & 2 & 0 & 0 \\
		0 & 0 & -1 & -2 & 0 & -3 & -1\\
		0 & 0 & 0 & 0 & 0 & 0 & 0\\
		0 & 0 & 0 & 0 & 0 & 6 & 2
		\end{bmatrix}
		\end{matrix}
		\]

\end{frame}

\begin{frame}{Exemple}

		\[
			\begin{matrix}[rcl]
		L_3  \leftrightarrow L_4
		&\Longrightarrow&
		\begin{bmatrix}[rrrrrr|r]
		\textcolor{red}{1} & 3 & -2 & 0 & 2 & 0 & 0 \\
		0 & 0 & \textcolor{red}{-1} & -2 & 0 & -3 & -1\\
		0 & 0 & 0 & 0 & 0 & \textcolor{red}{6} & 2\\
		0 & 0 & 0 & 0 & 0 & 0 & 0
		\end{bmatrix}
	\end{matrix}
	\]
	La matrice est dans une forme échelonnée. 
	Les variables correspondant aux pivots sont \textcolor{red}{$x_1, x_3$} et \textcolor{red}{$x_6$}: 
	ce sont les variables dépendantes.
	Les variables $x_2, x_4$ et $x_5$ sont les variables indépendantes, ou libres.
	\vfill
	
	Les variables indépendantes ou \textbf{libres} sont appelées ainsi parce qu'on peut choisir de leur
	attribuer n'importe quelle valeur. Un système avec
	des variables libres a une infinité de solutions.
\end{frame}

\begin{frame}{Équation paramétrique d'un cercle}
L'équation d'une courbe circulaire de rayon 1 centré à l'origine est
\[
x^2 + y^2 = 1
\]
On peut également écrire cette équation sous forme
paramétrique  
\[
\left\{
\begin{matrix}
x &=&\cos\theta \\
y &=& \sin\theta
\end{matrix}
\right.
\]
On a ici un seul paramètre, $\theta$, qui peut prendre
un nombre infini de valeurs. Bien que les points sur le cercle soient décrits habituellement par deux coordonnées,
$(x, y)$, cette courbe est une courbe à une dimension parce qu'on a un seul paramètre (ou variable libre).
\end{frame}

\begin{frame}{Équation paramétrique d'une droite}
Voici un exemple de l'équation d'une droite dans
le plan:
\[
y = \frac12 x + 3
\]
Habituellement, on dirait que la variable $x$ est une
variable \textbf{indépendante}, pouvant prendre une infinité
de valeurs, et que la variable $y$ est la variable \textbf{dépendante}.

On peut également écrire cette équation sous forme
paramétrique:
\[
\left\{
\begin{matrix}[lll]
x &=& t \\
y &=& \frac12 t + 3
\end{matrix}
\right.
\]

On a un seul paramètre, $t$, et la droite est donc
un objet à une seule dimension.
\end{frame}

\begin{frame}{Équation paramétrique d'une droite}
Une autre façon d'écrire une équation paramétrique
de la droite:
\[
y = \frac12 x + 3
\]
est:
\[
\left\{
\begin{matrix}[lll]
x &=& 2t \\
y &=& t + 3
\end{matrix}
\right.
\]
De cette façon, on n'a aucune fraction multipliant le paramètre $t$.
\end{frame}

\begin{frame}{Équation paramétrique d'une droite}
Dans le manuel, au lieu d'écrire
\[
\left\{
\begin{matrix}[lll]
x &=& t \\
y &=& \frac12 t + 3
\end{matrix}
\right.
\]
les auteurs préfèrent la notation suivante
\[
\left\{
\begin{matrix}[lll]
x &\mbox{quelconque} \\
y &= \frac12 x + 3
\end{matrix}
\right.
\]
Je recommande plutôt d'utiliser des variables pour
indiquer les paramètres et, si possible, de choisir
les paramètres pour éviter d'avoir des fractions
multipliant les paramètres.
\end{frame}


\begin{frame}{Équilibre des réactions chimiques}

L'eau est formée à partir de molécules d'hydrogène,
\cf{H2}, et d'oxygène, \cf{O2}, selon la réaction
chimique suivante:
\[
\ce{$2\,$H2 + O2 -> $2\,$H2O }
\]
Supposons que l'on ne connaisse pas à priori les proportions relatives de ces molécules et qu'on veuille utiliser les méthodes de l'algèbre linéaire pour les déterminer.
On écrirait
\[
\ce{$x\,$H2 + $y\,$O2 -> $z\,$H2O }
\]
En termes d'éléments individuels, ceci nous
donnerait les équations suivantes:
\[
\begin{matrix}[rrcl]
\cf{H}: & 2x &=& 2z \\
\cf{O}: & 2y &=& z
\end{matrix}
\]
\end{frame}

\begin{frame}{Équilibre des réactions chimiques}

On peut réécrire
\[
\begin{matrix}[rrcl]
\cf{H}: & 2x &=& 2z \\
\cf{O}: & 2y &=& z
\end{matrix}
\]
dans la forme habituelle\footnote{\color{red} On a déjà essentiellement
la forme requise pour une solution; je fais quand
même toutes les étapes requises pour un problème plus
compliqué pour démontrer la procédure.}:
\[
\left\{
\begin{matrix}
2x &&&-&2z&=& 0\\
&&2y&-&z&=&0
\end{matrix}
\right.
\]
correspondant à la matrice augmentée
\[
\begin{bmatrix}[rrr|r]
2 & 0 & -2 & 0 \\
0 & 2 & -1 & 0
\end{bmatrix}
\]
En divisant chacune des deux lignes par 2 on obtient la 
forme échelonnée réduite.
\end{frame}

\begin{frame}{Équilibre des réactions chimiques}
\[
\begin{bmatrix}[rrr|r]
1 & 0 & -1 & 0 \\
0 & 1 & -\frac12 & 0
\end{bmatrix}
\]
Ici, $x$ et $y$ sont les variables dépendantes et $z$ est la variable indépendante pouvant prendre n'importe quelle valeur.

La matrice augmentée ci-dessus correspond au système d'équations
\[
\left\{
\begin{matrix}
x &&&-&z&=& 0\\
&&y&-&\frac12 z&=&0
\end{matrix}
\right.
\]
et la solution générale est (dans la notation du livre)
\[
\left\{
\begin{matrix}[ll]
x & = z\\
y & = \frac12 z \\
z & \mbox{quelconque}
\end{matrix}
\right.
\]
\end{frame}

\begin{frame}{Équilibre des réactions chimiques}
Au lieu de la notation du livre, 
\[
\left\{
\begin{matrix}[ll]
x & = z\\
y & = \frac12 z \\
z & \mbox{quelconque}
\end{matrix}
\right.
\]
je préfère la notation paramétrique
\[
\left\{
\begin{matrix}[ll]
x & = t\\
y & = \frac12 t \\
z & = t
\end{matrix}
\right.
\]
... et, plus préférablement, avec seulement des
valeurs entières, que l'on peut obtenir facilement
en utilisant le paramètre $s$ tel que $t = 2s$
\[
\left\{
\begin{matrix}[ll]
x & = 2s\\
y & = s \\
z & = 2s
\end{matrix}
\right.
\]
\end{frame}

\begin{frame}{Équilibre des réactions chimiques}
En substituant ces valeurs dans l'équation de la
réaction chimique du départ, on obtient:
\[
\ce{$2{\color{red}s}\,$H2 + ${\color{red}s}\,$O2 -> $2{\color{red}s}\,$H2O }
\]
\vfill

\textbf{Important:} La convention pour écrire une réaction chimique équilibrée est d'utiliser les 
plus petits entiers possibles, et sans avoir de
fractions qui apparaissent comme coefficient pour
chaque molécule.   Ici, ceci correspond à choisir
$s=1$:
\[
\ce{$2\,$H2 + O2 -> $2\,$H2O }
\]
\end{frame}

\begin{frame}{Équilibre des réactions chimiques}
Quelques observations
\begin{itemize}
\item On a une infinité de solutions; pourquoi?
\item Ceci est un exemple de système d'équations linéaires \textbf{homogène} $\mat{A}\mat{x} = \mat{0}$.
Un tel système a toujours au moins une solution, soit
$\mat{x} = \mat{0}$.
\item La variable libre correspond au coefficient précédent \cf{H2O}; d'autres choix auraient été possibles.
\end{itemize}
\end{frame}

\begin{frame}{Équilibre des réactions chimiques}
Au lieu d'écrire:
\[
\left\{
\begin{matrix}
2x &&&-&2z&=& 0\\
&&2y&-&z&=&0
\end{matrix}
\right.
\]
supposons que l'on ait changé l'ordre des variables
et des équations
\[
\left\{
\begin{matrix}
-z &+& 2y &&& = &0\\
-2z&&&+&x&=&0
\end{matrix}
\right.
\]
correspondant à la matrice augmentée
\[
\begin{bmatrix}[rrr|r]
-1 & 2 & 0 & 0 \\
-2 & 0 & 2 & 0 
\end{bmatrix}
\]
Ceci n'est pas une forme échelonnée, contrairement
à ce qu'on avait auparavant. On peut multiplier
la première ligne par -1:
\end{frame}


\begin{frame}{Équilibre des réactions chimiques}
\[
\begin{bmatrix}[rrr|r]
1 & -2 & 0 & 0 \\
-2 & 0 & 2 & 0 
\end{bmatrix}
\]
puis faire $L_2 + 2L_1 \rightarrow L_2$
\[
\begin{bmatrix}[rrr|r]
1 & -2 & 0 & 0 \\
0 & -4 & 2 & 0 
\end{bmatrix}
\]
Si on divise la deuxième ligne par -4, on obtient la matrice
\[
\begin{bmatrix}[rrr|r]
1 & -2 & 0 & 0 \\
0 & 1 & -\frac12 & 0 
\end{bmatrix}
\]
\end{frame}


\begin{frame}{Équilibre des réactions chimiques}
Puis, on fait l'opération 
$L_1 + 2L_2 \rightarrow L_1$
\[
\begin{bmatrix}[rrr|r]
1 & 0 & -1 & 0 \\
0 & 1 & -\frac12 & 0 
\end{bmatrix}
\]
Ceci est la forme échelonnée réduite.
\footnote{Ici, la variable libre correspond à la variable de la troisième colonne qui est $x$ (le coefficient de \cf{H2}) et
non pas $z$ (le coefficient de \cf{H2O}) comme on avait auparavant.}
Ceci correspond au système suivant
\[
\left\{ \begin{matrix}
z && &-&x&=&0 \\
&&y&-&\frac12 x &=& 0
\end{matrix}
\right.
\qquad
{\color{red}\Rightarrow} \qquad
\left\{ \begin{matrix}[ll]
z &= x \\
y &= \frac12 x \\
x & \mbox{quelconque}
\end{matrix}
\right.
\]
\end{frame}


\begin{frame}{Équilibre des réactions chimiques}
Si on choisit d'utiliser le paramètre $s$ ($x = 2s$),
on obtient
\[
\left\{ \begin{matrix}
z &=& 2s \\
y &=& s \\
x &=& 2s
\end{matrix}
\right.
\]
qui est \textbf{exactement} ce qu'on avait auparavant.
\vfill

En utilisant des paramètres\footnote{Habituellement, on utilisera $r, s, t, \ldots$ pour dénoter les paramètres pouvant prendre n'importe quelle valeur.} pour les variables indépendantes, on traite toutes les variables sur
un pied d'égalité et la solution finale est toujours
la même, peut importe l'ordre qu'on choisit pour les
colonnes représentant les variables.
\end{frame}


\end{document}