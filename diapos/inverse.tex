%\documentclass[french, handout]{beamer}
\documentclass[french]{beamer}

\usetheme{Boadilla}
\usefonttheme{structuresmallcapsserif}

% % %  default French language settings
\usepackage[utf8]{inputenc}			% enable input of accented letters
\usepackage[T1]{fontenc}				% related to above
\usepackage[french]{babel}			% French language support

\usepackage{diapos}

%\usepackage{tikz}					% for drawing figures
%\usetikzlibrary{positioning}
%\tikzset{>=stealth}
%\newcommand{\tikzmark}[3][]{\tikz[overlay,remember picture,baseline] \node [anchor=base,#1](#2) {#3};}

	
\title{Inverse}
\author{Math 1253} % shows on bottom of each page by default; students do not care about
    % my name, nor the date in which this was written however, if a revision needs to 
    % be done, having a version number could be useful
\date{version 1}   % shows on bottom of page

\begin{document}
	\frame{\titlepage}
	
	\begin{frame}{Système d'équations linéaires}
	\begin{itemize}
	\item Questions?
	\item Inverse d'une matrice
	\item Transposée d'une matrice
	\item Matrices symétriques et anti-symétriques
	\end{itemize}
\end{frame}	

\begin{frame}{Définitions}
\begin{itemize}
\item Une \textbf{matrice carrée} est une matrice de taille $n\times n$, c'est-à-dire une matrice ayant un nombre de lignes égal au nombre de colonnes.
\item Une \textbf{matrice diagonale} est une matrice carré dont tous les éléments qui ne sont pas sur la diagonale,
c'est-à-dire les éléments de la forme $a_{ij}, i\ne j$, sont nuls; seuls les éléments sur la diagonale,
c'est-à-dire les éléments $a_{ii}$, peuvent être différents de zéro.
\end{itemize}


\[
\begin{pmatrix}
{\color{red}a_{11}} & 0 & 0 & \ldots & \ldots & 0 \\
0 & {\color{red}a_{22}} & 0 & \ldots & \ldots & 0\\
0 & 0 & {\color{red}\ddots} & 0 & \dots & \vdots\\
\vdots & \vdots& 0& {\color{red}a_{jj}} & 0& \vdots \\
\vdots & \vdots & 0 & 0 & {\color{red}\ddots}& 0 \\
 0 & \ldots &\ldots & 0 & 0 & {\color{red}a_{nn}}
\end{pmatrix}
\]
\end{frame}


\begin{frame}{Définitions}
\begin{itemize}
\item La \textbf{matrice identité} $\mat{I}_n$, est une matrice diagonale $n\times n$ dont tous les éléments sur la diagonale sont égaux à 1.  
\end{itemize}


\[
\begin{pmatrix}
{\color{red}1} & 0 & 0 & \ldots & \ldots & 0 \\
0 & {\color{red}1} & 0 & \ldots & \ldots & 0\\
0 & 0 & {\color{red}\ddots} & 0 & \dots & \vdots\\
\vdots & \vdots& 0& {\color{red}1} & 0& \vdots \\
\vdots & \vdots & 0 & 0 & {\color{red}\ddots}& 0 \\
 0 & \ldots &\ldots & 0 & 0 & {\color{red}1}
\end{pmatrix}
\]

Selon le contexte, on omet parfois l'indice et on écrit simplement $\mat{I}$ au lieu de
$\mat{I}_n$.
\end{frame}
	
\begin{frame}{Rappel sur les nombres: élément neutre et inverse}
	\begin{itemize}
	\item Le nombre $1$ est l'élément neutre de la multiplication: 
	\[1\cdot x = x \cdot 1 = x\]
	\item À l'exception de zéro, pour tout nombre réel, $x$, il existe un nombre $y$ appelé
	      inverse multiplicatif, c'est-à-dire que l'on a 
	      \[x\cdot y = y\cdot x = 1\] 
	      \begin{itemize}
	      \item On écrit habituellement l'inverse multiplicatif de $x$ par $x^{-1}$
	      \item Ceci respecte les règles de multiplication, $x^a \cdot x^b = x^{a+b}$
	      puisqu'on a 
	      \[x \cdot x^{-1} = x^{1 - 1} = x^0 = 1\]
	      \end{itemize}
	\end{itemize}
\end{frame}

\begin{frame}{Matrices: élément neutre de la multiplication}
	\begin{itemize}
	\item Une matrice identité est un élément neutre pour la multiplication.
	$\mat{A}_{m\times n} \mat{I}_n = \mat{I}_m \mat{A}_{m\times n} = \mat{A}_{m\times n}$
	\end{itemize}
	Dans ce qui suit, on se limite aux matrices carrées de taille $n\times n$.
\begin{itemize}
\item Une matrice carrée $\mat{A}$ peut \textbf{parfois} avoir un inverse, dénoté par 
$\mat{A}^{-1}$ tel que 
$\mat{A} \mat{A}^{-1} = \mat{I}$.
Si c'est le cas, on aura également
$\mat{A}^{-1} \mat{A} = \mat{I}$ 
\end{itemize}
\end{frame}

\begin{frame}{Exemple}
Exemple: trouvez, s'il existe, l'inverse de la matrice suivante.
\[
\matA = \begin{pmatrix}
1 & 2 & 3 \\
2 & 5 & 3 \\
1 & 0 & 8
\end{pmatrix}
\]
\end{frame}

\begin{frame}{Exemple}
Donc on cherche
\[
\matA^{-1} = \begin{pmatrix}
{\color{red}x_1} & {\color{blue}x_2} & {\color{brown}x_3} \\
{\color{red}y_1} & {\color{blue}y_2} & {\color{brown}y_3} \\
{\color{red}z_1} & {\color{blue}z_2} & {\color{brown}z_3}
\end{pmatrix}
\]
telle que
\[
\matA \matA^{-1} = \begin{pmatrix}
1 & 2 & 3 \\
2 & 5 & 3 \\
1 & 0 & 8
\end{pmatrix}\begin{pmatrix}
{\color{red}x_1} & {\color{blue}x_2} & {\color{brown}x_3} \\
{\color{red}y_1} & {\color{blue}y_2} & {\color{brown}y_3} \\
{\color{red}z_1} & {\color{blue}z_2} & {\color{brown}z_3}
\end{pmatrix}
=
\begin{pmatrix}
{\color{red}1} & {\color{blue}0} & {\color{brown}0} \\
{\color{red}0} & {\color{blue}1} & {\color{brown}0} \\
{\color{red}0} & {\color{blue}0} & {\color{brown}1}
\end{pmatrix} = \matI
\]
\end{frame}
\begin{frame}{Exemple}
Ceci est est équivalent à résoudre 3 systèmes d'équations:
\[
\begin{pmatrix}
1 & 2 & 3 \\
2 & 5 & 3 \\
1 & 0 & 8
\end{pmatrix}\begin{pmatrix}
{\color{red}x_1} \\
{\color{red}y_1} \\
{\color{red}z_1}
\end{pmatrix}
=
\begin{pmatrix}
{\color{red}1} \\
{\color{red}0} \\
{\color{red}0}
\end{pmatrix}
\]
\[
\begin{pmatrix}
1 & 2 & 3 \\
2 & 5 & 3 \\
1 & 0 & 8
\end{pmatrix}\begin{pmatrix}
 {\color{blue}x_2}  \\
 {\color{blue}y_2}  \\
 {\color{blue}z_2} 
\end{pmatrix}
=
\begin{pmatrix}
 {\color{blue}0}  \\
 {\color{blue}1}  \\
 {\color{blue}0} 
\end{pmatrix}
\]

\[
\begin{pmatrix}
1 & 2 & 3 \\
2 & 5 & 3 \\
1 & 0 & 8
\end{pmatrix}\begin{pmatrix}
 {\color{brown}x_3} \\
 {\color{brown}y_3} \\
 {\color{brown}z_3}
\end{pmatrix}
=
\begin{pmatrix}
 {\color{brown}0} \\
 {\color{brown}0} \\
 {\color{brown}1}
\end{pmatrix}
\]
\end{frame}
\begin{frame}{Exemple}
Les trois systèmes d'équations ont la même \textbf{forme} dans la notation de matrice augmentée:
\[
\begin{bmatrix}[ccc|c]
1 & 2 & 3 & a\\
2 & 5 & 3 & b\\
1 & 0 & 8 & c
\end{bmatrix}
\]
La solution sera obtenue à l'aide d'opérations élémentaires sur les lignes
pour obtenir
\[
\begin{bmatrix}[ccc|c]
1 & 0 & 0 & a'\\
0 & 1 & 0 & b'\\
0 & 0 & 1 & c'
\end{bmatrix}
\]
qu'on aurait à répéter 3 fois.  Au lieu de faire ceci, on peut combiner les
3 systèmes d'équations linéaires dans une seule matrice augmentée, et ne
faire les opérations élémentaires sur les lignes qu'une seule fois.
\end{frame}

\begin{frame}{Exemple}
On écrit: $[\matA|\matI]$
\[
\begin{bmatrix}[rrr|rrr]
1 & 2 & 3 & 1 & 0 & 0\\
2 & 5 & 3 & 0 & 1 & 0 \\
1 & 0 & 8 & 0 & 0 & 1
\end{bmatrix}
\]
\end{frame}

\begin{frame}
\[
\begin{matrix}[rcl]
    \begin{matrix}
    L_2 - 2 L_1 \rightarrow L_2 \\
    L_3 - L_1 \rightarrow L_3
    \end{matrix}
    &\Rightarrow&
    \begin{bmatrix}[rrr|rrr]
    1 & 2 & 3 & 1 & 0 & 0\\
    0 & 1 & -3 & -2 & 1 & 0 \\
    0 & -2 & 5 & -1 & 0 & 1
    \end{bmatrix}\\[20pt]
        L_3 + 2 L_2 \rightarrow L_3
    &\Rightarrow&
      \begin{bmatrix}[rrr|rrr]
    1 & 2 & 3 & 1 & 0 & 0\\
    0 & 1 & -3 & -2 & 1 & 0 \\
    0 & 0 & -1 & -5 & 2 & 1
    \end{bmatrix}\\[20pt]  
    -L_3 \rightarrow L_3
        &\Rightarrow&
      \begin{bmatrix}[rrr|rrr]
    1 & 2 & 3 & 1 & 0 & 0\\
    0 & 1 & -3 & -2 & 1 & 0 \\
    0 & 0 & 1 & 5 & -2 & -1
    \end{bmatrix}\\[20pt]    
    \begin{matrix}
    L_1 - 3L_3 \rightarrow L_1 \\
    L_2 + 3L_3 \rightarrow L_2
    \end{matrix}      
            &\Rightarrow&
      \begin{bmatrix}[rrr|rrr]
    1 & 2 & 0 & -14 & 6 & 3\\
    0 & 1 & 0 & 13 & -5 & -3 \\
    0 & 0 & 1 & 5 & -2 & -1
    \end{bmatrix}\\[20pt] 
     L_1 - 2 L_2 \rightarrow L_1
       &\Rightarrow&
      \begin{bmatrix}[rrr|rrr]
    1 & 0 & 0 & -40 & 16 & 9\\
    0 & 1 & 0 & 13 & -5 & -3 \\
    0 & 0 & 1 & 5 & -2 & -1
    \end{bmatrix}    
\end{matrix}
\]
\end{frame}

\begin{frame}{Exemple}
On a donc
\[
      \begin{bmatrix}[rrr|rrr]
    1 & 0 & 0 & -40 & 16 & 9\\
    0 & 1 & 0 & 13 & -5 & -3 \\
    0 & 0 & 1 & 5 & -2 & -1
    \end{bmatrix} 
\]
Le bloc de gauche est la matrice identité; le bloc de droite
est la matrice inverse de $\matA$
\[
\matA^{-1} = \begin{pmatrix}
-40 & 16 & 9 \\
13 & -5 & -3 \\
5 & -2 & -1
\end{pmatrix}
\]

\textbf{N.B.} Il faut utiliser la méthode de Gauss-Jordan; on ne peut
 pas utiliser l'élimination gaussienne.
\end{frame}

\begin{frame}{Interlude - matrice non carrée}
Soit:
\[
\matA = \begin{pmatrix}
-3 & 3 & -1 \\
-5 & 5 & 1
\end{pmatrix}
\qquad
\matB = \begin{pmatrix}
-3 & 2 & 0 \\
\frac32 & -\frac12 & 0
\end{pmatrix}
\qquad
\matC = \begin{pmatrix}
1 & 2\\
3 & 4\\
5 & 6
\end{pmatrix}
\]
On peut vérifier que, même si $\matA \neq \matB$, on a
$
\matA \matC = \matB \matC = \matI_2
$

\textbf{Important:} Les matrices $\matA$ et $\matB$ ne sont pas compatibles;
on ne peut pas les multiplier ensemble.
\end{frame}

\begin{frame}{Matrices carrées: l'inverse est unique}
Supposons que la matrice carrée $\matA$ a deux inverses, $\matB$ et $\matC$, et donc que
 $\matB\matA = \matI$ et $\matA\matC=\matI$. Nous allons démontrer que $\matB=\matC$, c'est-à-dire que l'inverse est unique. Nous avons
 \[
 \matB = \matB\matI = \matB(\matA\matC) = (\matB\matA)\matC =\matI\matC = \matC
 \]

\end{frame}

\begin{frame}{Propriétés des inverses pour les matrices carrées}
 Soit $\matA$ et $\matB$ deux matrices carrées de la même taille ($n\times n$) et $\matI$ la matrice identité correspondant. On pourrait démontrer que:
\begin{itemize}
\item  $\matI$ est inversible et $\matI^{-1} = \matI$.
\item Si $\matA$ est inversible, alors $\matA^{-1}$ est inversible et $(\matA^{-1})^{-1} = \matA$. Notez que ceci satisfait la loi des exposants.
\item Si $\matA$ et $\matB$ sont inversibles, alors $\matA\matB$ est inversible et $(\matA\matB)^{-1} = \matB^{-1}\matA^{-1}$
\item Si $\matA_1, \matA_2, \ldots, \matA_k$ sont inversibles, alors $\matA_1A_2 \ldots \matA_k$ est inversible et $(\matA_1A_2 \ldots \matA_k)^{-1} = \matA_k^{-1}\matA_{k-1}^{-1}\ldots \matA_2^{-1}\matA_1^{-1}$ est son inverse.
\item  Si $\matA$ est inversible, alors $\matA^k$ est également inversible et $(\matA^k)^{-1} = (\matA^{-1})^k$.
\item Si $\matA$ est inversible et que $c$ est un nombre
 différent de zéro, alors $c\matA$ est inversible et son
 inverse est $\displaystyle (c\matA)^{-1} = \frac{1}{c}\matA^{-1}$.
\end{itemize}  

\end{frame}

\begin{frame}{Inverse et systèmes d'équations linéaires}
Soit un système d'équations linéaires $\matA\mat{x} = \mat{b}$.

Si la matrice des coefficients, $\matA$, a un inverse, on peut multiplier par la gauche de chaque côté de cette équation par cet inverse et obtenir ce qui suit:
\[
{\color{red}\matA^{-1}} \matA \mat{x} =
{\color{red}\matA^{-1}} \mat{b}
\qquad \Rightarrow \qquad 
{\color{red}\matI} \mat{x} = 
{\color{red}\matA^{-1}} \mat{b}
\]
On a donc une solution unique:
\[
\mat{x} = \matA^{-1} \mat{b}
\]
Si l'inverse de la matrice des coefficients n'existe pas,
alors on a soit aucune solution ou une infinité de solutions.
\end{frame}

\begin{frame}{Transposée d'une matrice }

Soit une matrice $\matA_{m\times n}$; sa \textbf{transposée}, dénotée par $\transp{\matA}$,
est la matrice $n\times m$ obtenue en interchangeant les colonnes avec
les lignes de $\matA$.  Donc, si $\matA = [a_{ij}]$ alors $\transp{\matA} = [a_{ji}]$.

\begin{example}
    Quelle est la transposée de la matrice $\displaystyle 
    \matA = \begin{pmatrix}
        2 & 3 & 4 \\
        0 & 1 & 5
        \end{pmatrix}
    $

\textbf{Solution}
    \[
    \transp{\matA} = \begin{pmatrix}
                2 & 0\\
                3 & 1 \\
                4 & 5
        \end{pmatrix}
    \]
\end{example}



\end{frame}

\begin{frame}{Transposée d'une matrice }
    Soit $\matA=[a_{ij}]$ et $\matB = [b_{ij}]$ deux matrices $m\times n$, $\matC = [c_{ij}]$ une
    matrice carrée ($n\times n$) et $k$ un scalaire.  Alors:
\begin{itemize}
\item $\transp{(\transp{\matA})} = \matA$
\item $\transp{(\matA+\matB)} = \transp{\matA} + \transp{\matB}$
\item $ \transp{(k\matA)} = k \transp{\matA}$
\item Si $\matC$ est inversible, alors $\transp{\matC}$ est inversible et $(\transp{\matC})^{-1} = \transp{(\matC^{-1})}$.
\item $\transp{(\matA \transp{\matB})} = \transp{(\transp{\matB})} \transp{\matA} 
= \matB \transp{\matA}$. 
\item Si $m=n$ on a des matrices carrées et on pourrait alors écrire \[\transp{(\matA\matB)} = \transp{\matB}\transp{\matA}\]
ce qui la même forme que
\[(\matA\matB)^{-1} = \matB^{-1}\matA^{-1}\]
\end{itemize}

\end{frame}

\begin{frame}{Symétrie et anti-symétrie}
Pour les matrices carrées uniquement, on peut définir un concept de symétrie
ou d'anti-symétrie.
\begin{itemize}
\item Une matrice symétrique $\matA$ est une matrice qui est égale à sa transposée: $\matA = \transp{\matA}$.
\item Une matrice anti-symétrique $\matB$ est une matrice qui est égale à la 
négation de sa transposée: $\matB = - \transp{\matB}$
\begin{itemize}
\item Une matrice anti-symétrique doit avoir uniquement des zéros sur la
diagonale.
\end{itemize}
\end{itemize}
\end{frame}

\begin{frame}{Symétrie et anti-symétrie}
On peut démontrer que, pour n'importe quelle matrice carrée $\matM$, la matrice $\mat{S} = \frac{1}{2}(\matM + \transp{\matM})$ est symétrique.

\textbf{Démonstration:}
Tout d'abord, on se rappelle que la transposée d'une somme de matrices est égale 
à la somme des transposées. De plus, si on prend la transposée de la transposée d'une matrice, on retrouve
la matrice originale.  Utilisant ceci, nous avons:
\[
	\transp{\mat{S}} = \frac{1}{2}\transp{(\matM + \transp{\matM})} = \frac{1}{2}(\transp{\matM} + \matM) = \mat{S}
\]
	$\mat{S}$ est donc une matrice symétrique.
	
	On pourrait également démontrer de façon semblable que la matrice $\matA=\frac{1}{2}(\matM - \transp{\matM})$ est anti-symétrique.
\end{frame}

\begin{frame}{Symétrie et anti-symétrie}
On peut démontrer que, si $\matM$ est une matrice carrée, alors on peut écrire $\matM = \mat{S} + \matA$ où $\mat{S}$ est une matrice symétrique
et $\matA$ est une matrice anti-symétrique.



\textbf{Démonstration:}
 On vérifie facilement que, si on additionne $\matA$ et $\mat{S}$ tel que définis 
 de façon générale précédemment, on retrouve $\matM$.

\end{frame}

\begin{frame}{Symétrie et anti-symétrie}

\begin{example}

Voici un exemple concret :
\[
\matM = \begin{pmatrix}
2 & 4 \\ 6 & 8
\end{pmatrix}
\qquad\qquad
\transp{\matM} = \begin{pmatrix}
2 & 6 \\ 4 & 8
\end{pmatrix}
\]
\[
\mat{S} = \begin{pmatrix}
2 & 5 \\ 5 & 8
\end{pmatrix}
\qquad\qquad
\matA = \begin{pmatrix}
0 & -1 \\ 1 & 0
\end{pmatrix}
\qquad\qquad
\]

\end{example}
\end{frame}






\end{document}