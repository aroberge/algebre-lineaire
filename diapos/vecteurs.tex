\documentclass[french, handout]{beamer}
%\documentclass[french]{beamer}

\usepackage{hyperref}

\usetheme{Boadilla}
\usefonttheme{structuresmallcapsserif}

% % %  default French language settings
\usepackage[utf8]{inputenc}			% enable input of accented letters
\usepackage[T1]{fontenc}				% related to above
\usepackage[french]{babel}			% French language support

\usepackage{diapos}

%\usepackage{tikz}					% for drawing figures
%\usetikzlibrary{positioning}
%\tikzset{>=stealth}
%\newcommand{\tikzmark}[3][]{\tikz[overlay,remember picture,baseline] \node [anchor=base,#1](#2) {#3};}

	
\title{Les vecteurs}
\author{Math 1253} % shows on bottom of each page by default; students do not care about
    % my name, nor the date in which this was written however, if a revision needs to 
    % be done, having a version number could be useful
\date{version 1}   % shows on bottom of page

\begin{document}
	\frame{\titlepage}
	
	\begin{frame}{Les vecteurs}
	\begin{itemize}
	\item Questions?
	\item Démonstration sur \url{http://immersivemath.com}
	\begin{itemize}
	    \item Vecteur $\rightarrow$ flèche (élément de $\BBR^n$)
	    \item Addition
	    \item Soustraction
	    \item Multiplication par un scalaire
	    \item Propriétés de l'addition
	\end{itemize}
	\item Rappel: système d'équations linéaires
	\begin{itemize}
	    \item Forme traditionnelle
	    \item Interprétation graphique
	    \item Multiplication de matrices
	    \item Égalité de matrices
	\end{itemize}
	\item Nouveau: {\color{red}combinaison linéaire de vecteurs}
	\begin{itemize}
	    \item Interprétation graphique
	\end{itemize}
	\item Propriétés des vecteurs (chapitre 1)
	\item Propriétés des vecteurs (chapitre 4)
	\end{itemize}
\end{frame}	

\begin{frame}

\begin{itemize}
\item Question 2 (e): Si l'une des lignes d'une forme échelonnée d'une matrice complète est 
	$[\quad 0\quad 0\quad 0\quad 5\quad |\quad 0\quad ]$, alors le système associé est incompatible.\\ 
		\textit{\color{blue} Faux; ceci correspond à une équation comme $5x_4 = 0$ qui a une solution ($x_4=0$).}
\item Démonstration
\item Expliquer ce que sont $\BBR^2$, $\BBR^3$, $\BBR^n$.
\end{itemize}



\end{frame}
	
\begin{frame}{Système d'équations linéaires}
	Forme traditionnelle
	\begin{example}			
		\[
		\left\{
		\begin{matrix}
		    x&+&2y&=& 4 \\
		    x&-&y &=& 1
		\end{matrix}
		\right.
		\]		
			
		{\small\color{blue}Solution: $(x, y) = (2, 1)$}
	\end{example}		
	\end{frame}		


	\begin{frame}{Interprétation graphique}
		\begin{tikzpicture}
			% les axes
			\draw[->] (-1,0) -- (5,0) node[anchor=west]{\color{gray}x};
			\draw[->] (0,-1) -- (0,5) node[anchor=south]{\color{gray}y};
			% les deux droites
			\draw[-, thick, blue] (0,-1) -- (5,4) node[anchor=east]{\color{blue}\small $x-y=1$};
			\draw[-, thick, purple] (-1,2.5) --(5,-0.5) node[anchor= north]{\color{purple}\small $x+2y=4$};
			% le point d'intersection
			\coordinate (A) at (2,1);
			\draw[-] (2, -0.1) node[anchor=north]{\small 2} -- (2, 0.1);
			\draw[-] (-0.1, 1) node[anchor=east]{\small 1} -- (0.1, 1);
			\draw [dashed, gray] (2, 0.2) -- (A);
			\draw [dashed, gray] (0.2, 1) -- (A);
			\node [fill=black,inner sep=2pt,label=0:$\mbox{\quad(2,1)}$] at (2, 1) {};
		\end{tikzpicture}
	\end{frame}

   \begin{frame}{Rappel: multiplication de matrices}
    \[\matA\textbf{x} = \textbf{b}\]
	\[  
   \begin{pmatrix}
   1 & 2 \\
   1 & -1
   \end{pmatrix}
   \begin{pmatrix}
   x \\ y
   \end{pmatrix}
   =
   \begin{pmatrix}
  4 \\1
   \end{pmatrix}
   \]   
   \vfill
   Matrice augmentée:$\displaystyle\hspace{2cm}
   \begin{bmatrix}[rr|r]
   1 & 2 & 4\\
   1 & -1 & 1
   \end{bmatrix}
   $
   \end{frame}

    \begin{frame}{Rappel: Égalité de matrices}
		\[
\begin{pmatrix}
x + 2y\\
x - y
\end{pmatrix}		
		= \begin{pmatrix}
		4\\ 1
		\end{pmatrix}
		\qquad\Leftrightarrow\qquad
		\left\{
		\begin{matrix}
		    x&+&2y&=& 4 \\
		    x&-&y &=& 1
		\end{matrix}
		\right.
		\]	        
    \end{frame}
    


\begin{frame}{Nouveau: combinaison linéaire}
\[
	\begin{pmatrix}
x + 2y\\
x - y
\end{pmatrix}
=
\begin{pmatrix}
4 \\ 1
\end{pmatrix}
{\color{blue}\qquad\Leftrightarrow\qquad}
x \begin{pmatrix}
1 \\ 1
\end{pmatrix}
+ y \begin{pmatrix}
2 \\ -1
\end{pmatrix}
=
\begin{pmatrix}
4 \\ 1
\end{pmatrix}
	\]		


\[
\hspace*{4.5cm} x\,\mat{a}_1\quad +\quad y\,\mat{a}_2\quad = \quad\mat{b}
\]

\[
\hspace*{4.5cm} r\,\mat{a}_1\quad +\quad s\,\mat{a}_2\quad = \quad\mat{b}
\]
\end{frame}

\begin{frame}{Combinaison linéaire: interprétation graphique}
	\begin{tikzpicture}
	% les axes
	\draw[->] (-1,0) -- (6,0) node[anchor=west]{x};
	\draw[->] (0,-2) -- (0,4) node[anchor=south]{y};
	% labels and "grid"
	\draw[-] (1, -0.1) node[anchor=north]{\small 1} -- (1, 0.1);
	\draw[-] (2, -0.1) node[anchor=north]{\small 2} -- (2, 0.1);	
	\draw[-] (4, -0.1) node[anchor=north]{\small 4} -- (4, 0.1);
	\draw[-] (-0.1, 1) node[anchor=east]{\small 1} -- (0.1, 1);
	\draw[-] (-0.1, -1) node[anchor=east]{\small -1} -- (0.1, -1);
	\draw [dashed] (1, 0.2) -- (1, 1);
	\draw [dashed] (0.2, 1) -- (1, 1);
	\draw [dashed] (2, 0.2) -- (2, -1);
	\draw [dashed] (0.2, -1) -- (2, -1);
	\draw[->,thick,blue] (0,0) -- node[anchor=south]{$\mat{a}_1$} (1, 1);
	\draw[->,thick,red] (0,0) -- node[anchor=north]{$\mat{a}_2$} (2, -1);
	\end{tikzpicture}
\end{frame}

\begin{frame}{Combinaison linéaire: interprétation graphique {\tiny 2}}
	\begin{tikzpicture}
	% les axes
	\draw[->] (-1,0) -- (6,0) node[anchor=west]{x};
	\draw[->] (0,-2) -- (0,4) node[anchor=south]{y};
	% labels and "grid"
	\draw[-] (1, -0.1) node[anchor=north]{\small 1} -- (1, 0.1);
	\draw[-] (2, -0.1) node[anchor=north]{\small 2} -- (2, 0.1);	
	\draw[-] (4, -0.1) node[anchor=north]{\small 4} -- (4, 0.1);
	\draw[-] (-0.1, 1) node[anchor=east]{\small 1} -- (0.1, 1);
	\draw[-] (-0.1, -1) node[anchor=east]{\small -1} -- (0.1, -1);
	\draw [dashed] (4, 0.2) -- (4, 1);
	\draw [dashed] (0.2, 1) -- (4, 1);
	\draw[->,thick,blue] (0,0) -- node[anchor=south]{$\mat{a}_1$} (1, 1);
	\draw[->,thick,red] (0,0) -- node[anchor=north]{$\mat{a}_2$} (2, -1);
	\draw[->,thick,cyan] (0,0) -- node[anchor=north]{$\mat{b}$} (4, 1);
	\end{tikzpicture}
	On veut $r\mat{a}_1 + s\mat{a}_2 = \mat{b}$
\end{frame}

\begin{frame}{Combinaison linéaire: interprétation graphique {\tiny 3}}
	\begin{tikzpicture}
	% les axes
	\draw[->] (-1,0) -- (6,0) node[anchor=west]{x};
	\draw[->] (0,-2) -- (0,4) node[anchor=south]{y};
	% labels and "grid"
	\draw[-] (1, -0.1) node[anchor=north]{\small 1} -- (1, 0.1);
	\draw[-] (2, -0.1) node[anchor=north]{\small 2} -- (2, 0.1);	
	\draw[-] (4, -0.1) node[anchor=north]{\small 4} -- (4, 0.1);
	\draw[-] (-0.1, 1) node[anchor=east]{\small 1} -- (0.1, 1);
	\draw[-] (-0.1, -1) node[anchor=east]{\small -1} -- (0.1, -1);
	\draw[->,thick,blue] (0,0) -- node[anchor=south]{$\mat{a}_1$} (1, 1);
	\draw[->,thick,red] (0,0) -- node[anchor=north]{$\mat{a}_2$} (2, -1);
	\draw[->,thick,cyan] (0,0) -- node[anchor=north]{$\mat{b}$} (4, 1);
	\draw[->,thick,blue] (1,1) -- node[anchor=south]{$\mat{a}_1$} (2, 2);
	\draw[->,thick,red] (2, 2) -- node[anchor=north]{$\mat{a}_2$} (4, 1);
	\end{tikzpicture}
	Solution: $2\mat{a}_1 + \mat{a}_2 = \mat{b}$
\end{frame}


\begin{frame}{Combinaison linéaire}
\[
x \begin{pmatrix}
1 \\ 1
\end{pmatrix}
+ y \begin{pmatrix}
2 \\ -1
\end{pmatrix}
=
\begin{pmatrix}
4 \\ 1
\end{pmatrix}
	\]		
\[
x\mat{a}_1 + y\mat{a}_2 = \mat{b}
\]
\vfill
\[
\mat{a}_1x + \mat{a}_2y = \mat{b}
\]

\end{frame}

\begin{frame}{Combinaison linéaire}

\[
\mat{a}_1x + \mat{a}_2y = \mat{b}
\]
\vfill
Multiplication par blocs\footnote{Section 2.4 du manuel}
\[
[\mat{a}_1 \quad \mat{a}_2] \begin{bmatrix}
x\\y
\end{bmatrix}
= \mat{b}
\]\vfill
    \[\matA\textbf{x} = \textbf{b}\]

\end{frame}

\begin{frame}{Propriétés des vecteurs}
$\mat{u}$, $\mat{v}$, $\mat{w}$ sont trois vecteurs.\\
$\alpha$ et $\beta$ sont des scalaires.
\begin{enumerate}
\item $\mat{u} + \mat{v} = \mat{v} + \mat{u}$
\item $ (\mat{u} + \mat{v}) + \mat{w} = \mat{u} + (\mat{v} + \mat{w})$
\item $\mat{u} + \zero = \mat{u}$
\item $ \mat{u} + (-\mat{u}) = \zero$
\item $\alpha(\mat{u} + \mat{v}) = \alpha\mat{u} + \alpha\mat{v}$
\item $ (\alpha + \beta)\mat{u} = \alpha\mat{u} + \beta\mat{u}$
\item $\alpha(\beta\mat{u}) = (\alpha\beta)\mat{u}$
\item $1\mat{u} = \mat{u}$
\end{enumerate}
\end{frame}

\begin{frame}{Propriétés des vecteurs}
\begin{itemize}
\item Propriétés de $\BBR^n$ (chapitre 1, p:30)
\item Espace vectoriel (chapitre 4, p:206)
\end{itemize}
On observe les mêmes propriétés des "vecteurs".
\vfill À noter que l'on ne définit pas la multiplication
d'un vecteur par un autre vecteur
\end{frame}

\begin{frame}{Espace vectoriel}
Un \textbf{espace vectoriel} est un ensemble $V$ d'objets appelés \textit{vecteurs}, sur lesquels on définit
deux opérations, soit \textit{l'addition} ainsi que \textit{la multiplication par un scalaire}, et pour lequel
les axiomes suivant sont satisfaits pour tous les vecteurs $\mat{u}, \mat{v}, \mat{w}$ dans $V$
et pour tous les scalaires $\alpha, \beta$.
\begin{enumerate}
\item Fermeture sous l'addition: $\mat{u}+\mat{v}\in V$.
\item Commutativité de l'addition: $\mat{u} + \mat{v} = \mat{v} + \mat{u}$
\item Associativité de l'addition: $ (\mat{u} + \mat{v}) + \mat{w} = \mat{u} + (\mat{v} + \mat{w})$
\item Existence d'un élément neutre de l'addition: $\exists\zero\in V:  \mat{u} + \zero = \mat{u}$.
\item Existence d'un inverse additif: $\forall \mat{u}\in V\quad \exists-\mat{u}\in V: \mat{u} + (-\mat{u}) = \zero$.
\item Fermeture sous la multiplication: $\alpha\mat{u} \in V$.
\item Distributivité sur l'addition de vecteurs: $\alpha(\mat{u} + \mat{v}) = \alpha\mat{u} + \alpha\mat{v}$
\item Distributivité de l'addition de scalaires: $ (\alpha + \beta)\mat{u} = \alpha\mat{u} + \beta\mat{u}$
\item Associativité de la multiplication de scalaires: $\alpha(\beta\mat{u}) = (\alpha\beta)\mat{u}$
\item Élément neutre de la multiplication par un scalaire: $1\mat{u} = \mat{u}$
\end{enumerate}
\end{frame}

\begin{frame}{Exemple}

Articles sur Wikipédia
\begin{itemize}
\item \href{https://fr.wikipedia.org/wiki/Orbitale_atomique}{Orbitale atomique} : vecteurs
\item \href{https://fr.wikipedia.org/wiki/Combinaison_lin\%C3\%A9aire_d\%27orbitales_atomiques}{Combinaison linéaire d'orbitales atomiques}
\end{itemize}

\end{frame}


\end{document}
