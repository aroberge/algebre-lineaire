%\documentclass[french, handout]{beamer}
\documentclass[french]{beamer}

\usetheme{Boadilla}
\usefonttheme{structuresmallcapsserif}

% % %  default French language settings
\usepackage[utf8]{inputenc}			% enable input of accented letters
\usepackage[T1]{fontenc}				% related to above
\usepackage[french]{babel}			% French language support

\usepackage{diapos}

%\usepackage{tikz}					% for drawing figures
%\usetikzlibrary{positioning}
%\tikzset{>=stealth}
%\newcommand{\tikzmark}[3][]{\tikz[overlay,remember picture,baseline] \node [anchor=base,#1](#2) {#3};}

	
\title{Système d'équations linéaires}
\author{Math 1253} % shows on bottom of each page by default; students do not care about
    % my name, nor the date in which this was written however, if a revision needs to 
    % be done, having a version number could be useful
\date{version 1}   % shows on bottom of page

\begin{document}
	\frame{\titlepage}
	
	\begin{frame}{Système d'équations linéaires}
	\visible<2->{\color{red}2 équations (lignes)}
	\begin{example}
		\[
		\left\{
		\begin{matrix}
		    x &=& 2\visible<2->{\qquad\scriptstyle\color{red}{1}} \\
		    y&=& 1\visible<2->{\qquad\scriptstyle\color{red}{2}}
		\end{matrix}
		\right.
		\]
		\visible<3>{\small\color{blue}Solution: $(x, y) = (2, 1)$}
	\end{example}		
	\end{frame}

	\begin{frame}{Interprétation graphique}
		\begin{tikzpicture}
			% les axes
			\draw[->] (-1,0) -- (5,0) node[anchor=west]{\color{gray}x};
			\draw[->] (0,-1) -- (0,5) node[anchor=south]{\color{gray}y};
			% les deux droites
			\draw[-, thick, blue] (-1,1) -- (5,1) node[anchor=north]{\color{blue}\small $y=1$};
			\draw[-, thick, purple] (2,-1) --(2,5) node[anchor= east]{\color{purple}\small $x=2$};
			% le point d'intersection
			\coordinate (A) at (2,1);
			\draw[-] (2, -0.1) node[anchor=north]{\small 2} -- (2, 0.1);
			\draw[-] (-0.1, 1) node[anchor=east]{\small 1} -- (0.1, 1);
			\node [fill=black,inner sep=2pt,label=30:$\mbox{\quad(2,1)}$] at (2,1) {};
		\end{tikzpicture}
	\end{frame}

\begin{frame}{Opérations élémentaires sur les lignes}
On peut effectuer des opérations élémentaires sur les lignes pour obtenir un système d'équations
linéaires équivalent (c'est-à-dire ayant la même solution).

Il y a 3 type d'opérations élémentaires sur les lignes.
\end{frame}

	\begin{frame}{1. Échange de deux lignes}
	\begin{example}
		\[
		\left\{
		\begin{matrix}
		    x &=& 2{\qquad\scriptstyle\color{red}{1}} \\
		    y&=& 1{\qquad\scriptstyle\color{red}{2}}
		\end{matrix}
		\right.
		\]		
		
		\[
		L_1 \leftrightarrow L_2 \qquad \Rightarrow \qquad
		\left\{
		\begin{matrix}
		    y&=& 1{\qquad\scriptstyle\color{red}{1}}\\
		    x &=& 2{\qquad\color{red}{\scriptstyle 2}}
		\end{matrix}
		\right.
		\]		
		{\tiny\color{blue}Solution: $(x, y) = (2, 1)$}
	\end{example}		
	\end{frame}
	
	
	\begin{frame}{2. Remplacement ou combinaison linéaire}
	On remplace une ligne donnée par l'addition de celle-ci avec le multiple
	d'une autre ligne.
	\begin{example}
		\[
		\left\{
		\begin{matrix}
		    y&=& 1\\
		    x &=& 2
		\end{matrix}
		\right.
		\]		
		

		\[
		{{\color{red}L_2} - L_1 \rightarrow \color{red}L_2}  \qquad \Rightarrow \qquad
		\left\{
		\begin{matrix}
		    &&y&=& 1	\\	
		    x&-&y &=& 1
		\end{matrix}
		\right.
		\]		
			
		{\tiny\color{blue}Solution: $(x, y) = (2, 1)$}
	\end{example}		
	\end{frame}	
	
	
	\begin{frame}{3. Multiplication par un scalaire}
	On multiplie une ligne donnée par une constante (scalaire) différente de zéro.
	\begin{example}
		\[
		\left\{
		\begin{matrix}
		    &&y&=& 1	\\	
		    x&-&y &=& 1
		\end{matrix}
		\right.
		\]		
		
		\[
		{3{\color{red}L_1}\rightarrow \color{red}L_1}  \qquad \Rightarrow \qquad
		\left\{
		\begin{matrix}
		    &&3y&=& 3	\\	
		    x&-&y &=& 1
		\end{matrix}
		\right.
		\]		
			
		{\tiny\color{blue}Solution: $(x, y) = (2, 1)$}
	\end{example}		
	\end{frame}		
	
	\begin{frame}{2. Remplacement}
	On remplace une ligne donnée par l'addition de celle-ci avec le multiple
	d'une autre ligne.	\begin{example}
		\[
		\left\{
		\begin{matrix}
		    &&3y&=& 3	\\	
		    x&-&y &=& 1
		\end{matrix}
		\right.
		\]		
		
		\[
		{{\color{red}L_1} + 3 L_2 \rightarrow \color{red}L_1}  \qquad \Rightarrow \qquad
		\left\{
		\begin{matrix}
		    x&+&2y&=& 4 \\
		    x&-&y &=& 1
		\end{matrix}
		\right.
		\]		
			
		{\tiny\color{blue}Solution: $(x, y) = (2, 1)$}
	\end{example}		
	\end{frame}		


	\begin{frame}{Interprétation graphique}
		\begin{tikzpicture}
			% les axes
			\draw[->] (-1,0) -- (5,0) node[anchor=west]{\color{gray}x};
			\draw[->] (0,-1) -- (0,5) node[anchor=south]{\color{gray}y};
			% les deux droites
			\draw[-, thick, blue] (0,-1) -- (5,4) node[anchor=east]{\color{blue}\small $x-y=1$};
			\draw[-, thick, purple] (-1,2.5) --(5,-0.5) node[anchor= north]{\color{purple}\small $x+2y=4$};
			% le point d'intersection
			\coordinate (A) at (2,1);
			\draw[-] (2, -0.1) node[anchor=north]{\small 2} -- (2, 0.1);
			\draw[-] (-0.1, 1) node[anchor=east]{\small 1} -- (0.1, 1);
			\draw [dashed, gray] (2, 0.2) -- (A);
			\draw [dashed, gray] (0.2, 1) -- (A);
			\node [fill=black,inner sep=2pt,label=0:$\mbox{\quad(2,1)}$] at (2, 1) {};
		\end{tikzpicture}
	\end{frame}


\begin{frame}{Opération élémentaire sur les lignes}
\begin{itemize}
	\item on échange deux lignes :  $ L_i \leftrightarrow L_j$
	\item on remplace une ligne donnée par son multiple
	$ \alpha L_i  \quad \mbox{avec} \quad\alpha \ne 0$
	\item  on remplace une ligne donnée par l'addition de celle-ci avec le multiple d'une autre ligne
	$L_i + \beta L_j \rightarrow L_i$
\end{itemize}
\visible<2->{
À noter qu'à chaque étape, on peut transformer plus qu'une ligne sujet aux conditions suivantes:
\begin{itemize}
	\item on ne fait pas plus qu'une transformation pour une ligne donnée;
	\item on n'utilise pas une ligne sur laquelle on fait une transformation.
\end{itemize}
}
\end{frame}

	\begin{frame}{Notation matricielle}
		\begin{itemize}
		\item Notation habituellement plus compacte
		\item Plus utile pour les démonstrations (preuves)
		\end{itemize}
	
	\end{frame}

	\begin{frame}{Notation matricielle}
		\begin{block}{}
					\[
				\tikzmark{Amatrix}{A}\quad \tikzmark{xvec}{\textbf{x}}\quad=\quad 
				\tikzmark{bvec}{\textbf{b}}
			\]
			\begin{tikzpicture}[overlay, remember picture,node distance =1.5cm]
				\node (Amatrixdesc) [below left=of Amatrix ]{Matrice des coefficients};
				\draw[,->,thick] (Amatrixdesc) to [in=-90,out=90] (Amatrix);
				\node[red] (xvecdesc) [below =of xvec]{vecteur inconnu};
				\draw[red,->,thick] (xvecdesc) to [in=-90,out=90] (xvec);
				\node[blue] (bvecdesc) [below right =of bvec]{vecteur constant};
				\draw[blue,->,thick] (bvecdesc) to [in=-90,out=90] (bvec);
			\end{tikzpicture}
			\vspace*{2cm}
		\end{block}
		\begin{block}{Matrice augmentée}
		$$[\matA | \textbf{b}]$$
		\end{block}
	\end{frame}

    \begin{frame}{Rappel: Égalité de matrices}
		\[
\begin{pmatrix}
x + 2y\\
x - y
\end{pmatrix}		
		= \begin{pmatrix}
		4\\ 1
		\end{pmatrix}
		\qquad\Leftrightarrow\qquad
		\left\{
		\begin{matrix}
		    x&+&2y&=& 4 \\
		    x&-&y &=& 1
		\end{matrix}
		\right.
		\]	        
    \end{frame}
    
   \begin{frame}{Multiplication de matrices\footnote{Rappel: taille}}
   \begin{block}{}
   \[
   \begin{pmatrix}
   a & b \\
   c & d
   \end{pmatrix}
   \begin{pmatrix}
   x \\ y
   \end{pmatrix}
   =
   \begin{pmatrix}
   ax + by \\
   cx + dy
   \end{pmatrix}
   \]
   \end{block}
   \begin{example}<2->
   \[
   \begin{pmatrix}
   1 & 2 \\
   1 & -1
   \end{pmatrix}
   \begin{pmatrix}
   x \\ y
   \end{pmatrix}
   =
   \begin{pmatrix}
   x + 2y \\
   x - y
   \end{pmatrix}
   \]
   \visible<3->{%
      \[
\begin{pmatrix}
x + 2y\\
x - y
\end{pmatrix}		
		= \begin{pmatrix}
		4\\ 1
		\end{pmatrix}
		\qquad\Leftrightarrow\qquad
		\left\{
		\begin{matrix}
		    x&+&2y&=& 4 \\
		    x&-&y &=& 1
		\end{matrix}
		\right.
		\]
		}   \visible<4->{%
		\[  
   {\color{blue}\Rightarrow}
   \begin{pmatrix}
   1 & 2 \\
   1 & -1
   \end{pmatrix}
   \begin{pmatrix}
   x \\ y
   \end{pmatrix}
   =
   \begin{pmatrix}
  4 \\1
   \end{pmatrix}
   \]   
   \[\matA\textbf{x} = \textbf{b}\]
   }
   \visible<5>{%
   Matrice augmentée:$\displaystyle\hspace{2cm}
   \begin{bmatrix}[rr|r]
   1 & 2 & 4\\
   1 & -1 & 1
   \end{bmatrix}
   $
   }
   \end{example}   
   \end{frame}

\begin{frame}{Solution avec la matrice augmentée}
\begin{example}
\[
   \begin{bmatrix}[rr|r]
   1 & 2 & 4\\
   1 & -1 & 1
   \end{bmatrix}
\]
   \visible<2->{$L_2 - L_1 \rightarrow L_2$}
   \visible<3->{\[
   \begin{bmatrix}[rr|r]
   1 & 2 & 4\\
   0 & -3 & -3
   \end{bmatrix}   
   \]}
\end{example}
\end{frame}

\begin{frame}{Solution avec la matrice augmentée}
\begin{example}
\[
   \begin{bmatrix}[rr|r]
   1 & 2 & 4\\
   0 & -3 & -3
   \end{bmatrix} 
\]
   \visible<2->{$L_2\rightarrow -\frac13 L_2$}
   \visible<3->{\[
   \begin{bmatrix}[rr|r]
   1 & 2 & 4\\
   0 & 1 & 1
   \end{bmatrix}   
   \]}
\end{example}
\end{frame}


\begin{frame}{Solution avec la matrice augmentée}
\begin{example}
\[
   \begin{bmatrix}[rr|r]
   1 & 2 & 4\\
   0 & 1 & 1
   \end{bmatrix} 
\]
   \visible<2->{$L_1 - 2L_2 \rightarrow L_1$}
   \visible<3->{\[
   \begin{bmatrix}[rr|r]
   1 & 0 & 2\\
   0 & 1 & 1
   \end{bmatrix}   
   \]}
\visible<4->{\color{blue}\[
		\left\{
		\begin{matrix}
		    x&\color{lightgray}+&\color{lightgray}0y&=& 2	\\	
		    \color{lightgray}0x&\color{lightgray}+&y &=& 1
		\end{matrix}
		\right.		
\]
		{\color{blue}\tiny Solution: $(x, y) = (2, 1)$}
}   
   
\end{example}
\end{frame}

\begin{frame}{Exemple avec 3 équations}
\[
	\left\{
	\begin{matrix}
	x &+& 2y &-& z &=& -3\\
	2x &+& y &+& z &=& 6\\
	3x &+& 4y &+& z &=& 5
	\end{matrix}\right.
\]
\visible<2>{%
\[
\begin{bmatrix}[rrr|r]
1 & 2 & -1 & -3 \\
2 & 1 & 1 & 6 \\
3 & 4 & 1 & 5
\end{bmatrix}
\]
}
\end{frame}

\begin{frame}{Exemple avec 3 équations}
\[
L_2 -2L_1 \rightarrow L_2 \Rightarrow
\color{red}
\left\{
	\begin{matrix}
	x &+& 2y &-& z &=& -3\\
	 &-& 3y &+& 3z &=& 12\\
	3x &+& 4y &+& z &=& 5
	\end{matrix}\right.
	\qquad\color{black}
	\begin{bmatrix}[rrr|r]
1 & 2 & -1 & -3 \\
0 & -3 & 3 & 12 \\
3 & 4 & 1 & 5
\end{bmatrix}
\]
\end{frame}

\begin{frame}{Exemple avec 3 équations}
\[
L_3 -3L_1 \rightarrow L_3 \Rightarrow
\color{red}
\left\{
	\begin{matrix}
	x &+& 2y &-& z &=& -3\\
	&-& 3y &+& 3z &=& 12\\
	&-& 2y &+& 4z &=& 14
	\end{matrix}\right.
	\qquad\color{black}
	\begin{bmatrix}[rrr|r]
1 & 2 & -1 & -3 \\
0 & -3 & 3 & 12 \\
0 & -2 & 4 & 14
\end{bmatrix}
\]
\end{frame}

\begin{frame}{Exemple avec 3 équations}
\[
\begin{matrix}
\frac{-1}{3}L_2 \rightarrow L_2 \\
\\
\frac{1}{2}L_3 \rightarrow L_3
\end{matrix}
 \Rightarrow
\color{red}
\left\{
	\begin{matrix}
	x &+& 2y &-& z &=& -3\\
	&& y &-& z &=& -4\\
	&-& y &+& 2z &=& 7
	\end{matrix}\right.
	\qquad\color{black}
	\begin{bmatrix}[rrr|r]
1 & 2 & -1 & -3 \\
0 & 1 & -1 & -4 \\
0 & -1 & 2 & 7
\end{bmatrix}
\]
\end{frame}

\begin{frame}{Exemple avec 3 équations}
\[
L_3 + L_2 \rightarrow L_3 \Rightarrow
 \Rightarrow
\color{red}
\left\{
	\begin{matrix}
	x &+& 2y &-& z &=& -3\\
	&& y &-& z &=& -4\\
	&&  && z &=& 3
	\end{matrix}\right.
	\qquad\color{black}
	\begin{bmatrix}[rrr|r]
1 & 2 & -1 & -3 \\
0 & 1 & -1 & -4 \\
0 & 0 & 1 & 3
\end{bmatrix}
\]
Ceci est la forme échelonnée.
\end{frame}

\begin{frame}{Exemple avec 3 équations}
\[
		\begin{matrix}
		L_1 + L_3 \rightarrow L_1\\
		L_2 + L_3 \rightarrow L_2\\
		\end{matrix}
 \Rightarrow
\color{red}
\left\{
	\begin{matrix}
		x &+& 2y && &=& 0\\
		 && y && &=& -1\\
		 &&  && z &=& 3
	\end{matrix}\right.
	\qquad\color{black}
	\begin{bmatrix}[rrr|r]
        1 & 2 & 0 & 0 \\
        0 & 1 & 0 & -1 \\
        0 & 0 & 1 & 3
\end{bmatrix}
\]
\end{frame}

\begin{frame}{Exemple avec 3 équations}
\[
	L_1 -2 L_2 \rightarrow L_1
 \Rightarrow
\color{red}
\left\{
	\begin{matrix}
	x & && =& 2\\
	 & y &&=& -1\\
	  && z &=& 3
	\end{matrix}\right.
	\qquad\color{black}
	\begin{bmatrix}[rrr|r]
        1 & 0 & 0 & 2 \\
        0 & 1 & 0 & -1 \\
        0 & 0 & 1 & 3
\end{bmatrix}
\]
Solution: $(x, y, z) = (2, -1, 3)$.  [Vérifier.]

\color{blue}
\[
	\left\{
	\begin{matrix}
	x &+& 2y &-& z &=& -3\\
	2x &+& y &+& z &=& 6\\
	3x &+& 4y &+& z &=& 5
	\end{matrix}\right.
	\]

\end{frame}

\end{document}