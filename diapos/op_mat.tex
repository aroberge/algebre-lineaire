%\documentclass[french, handout]{beamer}
\documentclass[french]{beamer}

\usepackage{diapos}

\title{Opérations matricielles}
\author{Math 1253} % shows on bottom of each page by default; students do not care about
	% my name, nor the date in which this was written however, if a revision needs to 
	% be done, having a version number could be useful
\date{version 1}   % shows on bottom of page

\begin{document}
	\frame{\titlepage}
	
	\begin{frame}
	\begin{block}{Matrice}
	Soit $m$ et $n$ deux entier positifs; une matrice de taille\footnote{Au lieu de \textbf{taille}, 
		certains utilisent parfois le mot \textbf{dimension}.  
		Cependant, le mot dimension peut également désigner une autre caractéristique importante en algèbre linéaire et,
		pour cette raison, nous n'utilisons pas le mot dimension comme synonyme de taille. 
		À noter que \textit{taille $m\times n$} se lit \textit{taille m par n}.} $m\times n$
	est une collection de $mn$ nombres arrangés dans un tableau rectangulaire:
	\[
	\begin{matrix}
	&\text{$n$ colonnes} \\
	\text{$m$ lignes}& \begin{bmatrix}
	a_{11} & a_{12} & \ldots & a_{1n}\\
	a_{21} & a_{22} & \ldots & a_{2n}\\
	\vdots & \vdots & \vdots & \vdots \\
	a_{m1} & a_{m2} & \ldots & a_{mn}
	\end{bmatrix}
	\end{matrix}
	\]
	\end{block}
	\end{frame}
	
	\begin{frame}
		\begin{block}{Coefficient}
		$a_{ij}$ est le coefficient qui apparait dans la ligne $i$ et dans la colonne $j$:
		\[
		\begin{matrix}
		j\hspace*{1.1cm} \\
		\begin{matrix}\\ \\ i \\ \\ \\ \\ \\ \end{matrix}
		\begin{pmatrix}
		&&&\cdot&&&&&& \\
		\cdot&\cdot&\cdot&a_{ij}&\cdot&\cdot&\cdot&\cdot&\cdot&\cdot\\
		&&&\cdot&&&&&&\\
		&&&\cdot&&&&&&\\
		&&&\cdot&&&&&&
		\end{pmatrix}
		\end{matrix}
		\]
		Pour une matrice $A$, on écrira parfois $(A)_{ij}$ au lieu de $a_{ij}$ pour dénoter ce coefficient.
	\end{block}
	Note: on peut utiliser soit des crochets, $[...]$, ou des parenthèses, $(...)$, autour des coefficients d'une matrice : le choix de l'un ou de l'autre est sans importance.
	\end{frame}

	\begin{frame}{}
		\begin{block}{Notation}
		\begin{itemize}
			\item Matrice carrée: taille $n \times n$
			\item Coefficient diagonaux: $a_{ii}$
			\item Diagonale principale: tous les coefficients diagonaux 
			\item Matrice nulle $O$: matrice dont tous les coefficients sont nuls.
			\item Vecteur\footnote{On va surtout utiliser des matrices carrées et des vecteurs.} (ou vecteur colonne): matrice de taille $n \times 1$
		\end{itemize}
		\end{block}		

	\end{frame}


\end{document}