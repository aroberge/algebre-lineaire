\documentclass[french, handout]{beamer}
%\documentclass[french]{beamer}

\usepackage{diapos}

\title{Les matrices et les opérations de base}
\author{Math 1253} % shows on bottom of each page by default; students do not care about
	% my name, nor the date in which this was written however, if a revision needs to 
	% be done, having a version number could be useful
\date{version 1}   % shows on bottom of page

\begin{document}
	\frame{\titlepage}
	
\begin{frame}
\begin{itemize}
\item Résultat du sondage
\item Brève revue du premier devoir
\item Propriétés des nombres
\item Matrices: taille, coefficient, égalité
\item Opérations: addition, soustraction, etc.
\end{itemize}
\end{frame}
	
\begin{frame}{Rappel: propriétés des entiers}
Soit $a, b, c$ trois entiers (par exemple: 2, 3, 5).
\begin{itemize}[<+-|alert@+>]
\item $a+b = b+a$ \qquad \explain{commutativité de l'addition}
\item $(a+b) + c = a + (b+c)$ \qquad \explain{associativité de l'addition}
\item $ab = ba$ \qquad \explain{commutativité de la multiplication}
\item $(ab)c = a(bc)$ \qquad \explain{associativité de la multiplication}
\item $a + \textcolor{blue}{0} = a$ \qquad \explain{élément neutre de l'addition}
\item $a + (-a) = 0$ \explain{inverse additif - pour les entiers relatifs}
\item $\textcolor{blue}{1}a = a$ \qquad \explain{élément neutre de la multiplication}
\item $a(b+c) = ab + ac$ \qquad \explain{distributivité de la multiplication sur l'addition}
\item $\left(a^b\right)^c = a^{bc}$ \qquad \explain{puissance d'une puissance}
\item $a^b a^c = a^{b+c}$ \qquad \explain{produit des puissances}
\end{itemize}

\end{frame}	

\begin{frame}{Rappel: propriétés des nombres réels}
Soit $a, b, c$ trois nombres réels. Les réels ont les mêmes propriétés que les entiers;
ils ont également la propriété additionnelle suivante
\begin{itemize}
\item $a a^{-1} = \textcolor{red}{1}$ \explain{inverse multiplicatif}
\item<2-> Sauf pour 1 et -1, l'inverse multiplicatif d'un entier n'est pas un entier.
\item<2-> L'élément neutre de l'addition (0) n'a pas d'inverse multiplicatif.
\end{itemize}
\begin{block}{Questions?}<3>
\end{block}

\end{frame}
	
	\begin{frame}{Matrice}
	Une matrice est une collection de nombres arrangés dans un tableau rectangulaire
		\[
\mat{A} = \begin{pmatrix}
1 & 2 \\
3 & 0 
\end{pmatrix} \qquad 
\mat{B} = \begin{bmatrix}
1 & 0 & 9 \\
2 & 1 & 0 
\end{bmatrix} \qquad
\mat{C} = \begin{pmatrix}
2 & 1 & 3 & 4\\
1 & 2 & 3 & 4
\end{pmatrix} \]
\[
\mat{D} = \begin{bmatrix}
1 & 0  \\
0 & 1  \\
1 & 1
\end{bmatrix} \qquad
\mat{E} = \begin{pmatrix}
1 & 1 & 1 \\
2 & 2 & 2 \\
3 & 3 & 3
\end{pmatrix} \qquad
	\]
\end{frame}		
	
	\begin{frame}{Matrice}
	Soit $m$ et $n$ deux entier positifs; une matrice de taille\footnote{Au lieu de \textbf{taille}, 
		certains utilisent parfois le mot \textbf{dimension}.  
		Cependant, le mot dimension peut également désigner une autre caractéristique importante en algèbre linéaire et,
		pour cette raison, nous n'utilisons pas le mot dimension comme synonyme de taille. 
		À noter que \textit{taille $m\times n$} se lit \textit{taille m par n}.} $m\times n$
	est une collection de $mn$ nombres arrangés dans un tableau rectangulaire:
\[
\begin{matrix}[cc]
&\text{$n$ colonnes} \\
\text{$m$ lignes}& \begin{pmatrix}
a_{11} & a_{12} & \ldots & a_{1n}\\
a_{21} & a_{22} & \ldots & a_{2n}\\
\vdots & \vdots & \vdots & \vdots \\
a_{m1} & a_{m2} & \ldots & a_{mn}
\end{pmatrix}
\end{matrix}
\]
	\end{frame}
	
	\begin{frame}{Taille d'une matrice}
	On indique parfois la taille des matrices en ajoutant un indice de la forme\\
	\textbf{nombre de lignes} $\times$ \textbf{nombre de colonnes}.
		\[
\mat{A}_{\color{red}2\times2} = \begin{pmatrix}
1 & 2 \\
3 & 0 
\end{pmatrix} \qquad 
\mat{B}_{2\times3} = \begin{pmatrix}
1 & 0 & 9 \\
2 & 1 & 0 
\end{pmatrix} \qquad
\mat{C}_{2\times4} = \begin{pmatrix}
2 & 1 & 3 & 4\\
1 & 2 & 3 & 4
\end{pmatrix} \]
\[
\mat{D}_{3\times2} = \begin{pmatrix}
1 & 0  \\
0 & 1  \\
1 & 1
\end{pmatrix} \qquad
\mat{E}_{3\times3} = \begin{pmatrix}
1 & 1 & 1 \\
2 & 2 & 2 \\
3 & 3 & 3
\end{pmatrix} \qquad
	\]
\end{frame}		
		
\begin{frame}{Vecteur dans $\BBR^n$}
Une matrice (colonne) de taille $n\times 1$ est appelée un \textbf{vecteur} de $\BBR^n$.

\[
\begin{pmatrix}[c] x\\y
\end{pmatrix}
\]

$\BBR$: l'axe des réels

$\BBR^2$: le plan

$\BBR^3$: l'espace (à trois dimensions).

\vfill
N.B. On utilisera plus tard le mot \textbf{vecteur} pour désigner d'autres types d'objets.

\end{frame}
	
	\begin{frame}
		\begin{block}{Coefficient (ou élément d'une matrice)}
		$a_{ij}$ est le coefficient qui apparait dans la ligne $i$ et dans la colonne $j$:
\[
\begin{matrix}[c]
j\hspace*{1.1cm} \\
\begin{matrix}[c]\\ \\ i \\ \\ \\ \\ \\ \end{matrix}
\begin{pmatrix}[cccccccccc]
&&&\cdot&&&&&& \\
\cdot&\cdot&\cdot&a_{ij}&\cdot&\cdot&\cdot&\cdot&\cdot&\cdot\\
&&&\cdot&&&&&&\\
&&&\cdot&&&&&&\\
&&&\cdot&&&&&&
\end{pmatrix}
\end{matrix}
\]
		Pour une matrice $A$, on écrira parfois $(A)_{ij}$ au lieu de $a_{ij}$ pour dénoter ce coefficient.
	\end{block}
	Note: on peut utiliser soit des crochets, $[...]$, ou des parenthèses, $(...)$, autour des coefficients d'une matrice : le choix de l'un ou de l'autre est sans importance. Également, on utilise une lettre majuscule pour la matrice, et la même lettre minuscule pour ses coefficients.
	\end{frame}

	\begin{frame}{Coefficient (ou élément d'une matrice)}
	Soit
		\[
\mat{C}_{2\times4} = \begin{bmatrix}
2 & 1 & 7 & 4\\
1 & 2 & {\color{red}3} & 4
\end{bmatrix} \]
On a $c_{23} = (\mat{C})_{23} = 3$.
\begin{block}{Questions?}<2>
\end{block}
\end{frame}		

\begin{frame}{Égalité}
On dit de deux matrices qu'elles sont égales si elles ont la même taille
et que leur coefficients sont égaux deux à deux, c'est-à-dire\footnote{Le symbole $\forall$ veut dire 
``quoi que ce soit'' ou ``pour tout''.}
\[
\matA = \matB \quad {\color{red}\Longleftrightarrow} \quad [a_{ij}] = [b_{ij}]  \quad \forall i, j
\]

\begin{example}
Soient les matrices
    \[
    \matA = \begin{pmatrix}
        2 & 1 \\
        3 & 4
        \end{pmatrix}
    \qquad
    \matB = \begin{pmatrix}
        2 & x \\
        3 & 4
        \end{pmatrix}
        \qquad
    \matC = \begin{pmatrix}
        x & 1 \\
        3 & x
        \end{pmatrix}
    \]
Avec un choix approprié pour la variable $x$, est-il possible que $\matA =\matB$?
Est-il possible que $\matA = \matC$?
\end{example}
Réponse: ?
\end{frame}		
		
	\begin{frame}{Résumé des opérations de base}
	\begin{block}{}<2>
\begin{itemize}
	\item Addition de deux matrices: $\matA + \matB = \matC$
	\item Multiplication par un scalaire (nombre): ${\color{red}c}\matA = \matB$
	\begin{itemize}
		\item La soustraction de deux matrices est un combinaison de
		deux opérations:  $\matA - \matB = \matA + ({\color{red}-1}\matC)$
	\end{itemize}
	\item Multiplication de deux matrices: $\matA\matB = \matC$
	\item IMPORTANT: On ne peut pas diviser deux matrices.
\end{itemize}
	\end{block}
\end{frame}

\begin{frame}{Addition}
Si \textbf{et seulement si deux matrices}, $\matA$ et $\matB$, sont de la même taille, alors il est possible
de les additionner.  
\begin{definition}
Soit, deux matrices, $\matA$ et $\matB$ ayant la même taille.  L'addition de
ces matrices est une matrice $\matC$ de la même taille dont les coefficients sont
la somme des coefficients correspondants des matrices $\matA$ et $\matB$.
\[
\matA + \matB = \matC \quad {\color{red}\Longleftrightarrow} \quad [a_{ij}] + [b_{ij}] = [a_{ij} + b_{ij}] = [c_{ij}]
\]
\end{definition}
\end{frame}

\begin{frame}{Addition}
\begin{example}
    Soient les matrices \[
    \matA = \begin{pmatrix}
        2 & 1 \\
        3 & 4
        \end{pmatrix}
    \qquad
    \matB = \begin{pmatrix}
        2 & 1 \\
        3 & 5
        \end{pmatrix}
        \qquad
    \matC = \begin{pmatrix}
        2 & 1 & 0\\
        3 & 4 & 0
        \end{pmatrix}
    \]
    Calculez, si possible, $\matA+\matB, \matA+\matC$ et $\matB+\matC$.
\end{example}
\begin{block}{Solution}<2->
    Nous avons $\displaystyle
    \matA+\matB = \begin{pmatrix}
                {\color{red}2} & 1\\
                3 & 4
                \end{pmatrix} + 
                \begin{pmatrix}
                {\color{red}2} & 1\\
                3 & 5
                \end{pmatrix} = \begin{pmatrix}
                {\color{red}2+2} & 1+1\\
                3+3 & 4+5
                \end{pmatrix} = \begin{pmatrix}
            {\color{red}4} & 2\\
            6 & 9
            \end{pmatrix}
    $.\\[5pt]   
    Les sommes $\matA+\matC$ et $\matB+\matC$ sont indéfinies parce que les matrices ne
    sont pas de la même taille.
    \end{block}
\end{frame}

\begin{frame}{Multiplication par un scalaire}
Par \textbf{scalaire}, on entend un nombre arbitraire, qui sera
habituellement un réel ou qui pourrait être un nombre complexe (selon
le contexte).   
\begin{definition}
Lorsqu'une matrice $\matA$ est multipliée par un scalaire $c$,
la matrice résultante est telle que chaque coefficient est multiplié par $c$
\[
c(a_{ij}) = (ca_{ij})
\]
\end{definition}

À noter que l'on écrit habituellement $(-1) \matA = -\matA$.
\end{frame}

\begin{frame}{Multiplication par un scalaire}
\begin{example}
    Soit la matrice \[
    \matC = \begin{pmatrix}
        2 & 1 & 0\\
        3 & 4 & 0
        \end{pmatrix}
    \]
    Calculez $2\matC$.
\end{example}
\begin{block}{Solution}<2>
\[
    2\matC = \begin{pmatrix}
        4 & 2 & 0\\
        6 & 8 & 0
        \end{pmatrix}
    \]
    \end{block}
\begin{block}{Questions?}<3>
\end{block}   
\end{frame}

\begin{frame}{Multiplication par un scalaire}
\begin{example}
	Soit $\matM$ une matrice $m\times n$ et $c$ un scalaire.  Démontrez que si $c\matM = \zero$, alors
	soit $c=0$ ou $\matM=\zero$.
\end{example}
\begin{block}{Solution}<2->
	Écrivons $\matM=[m_{ij}]$; par conséquent, $c\matM = [cm_{ij}]$.  Supposons que $c\matM = \zero$.  Ceci implique que,
	$ cm_{ij}=0\,\forall i, j $.  Ceci est vrai si $c=0$ ou que tous les $m_{ij}$ sont égaux à zéro; dans ce dernier
	cas, nous aurions $\matM=\zero$.
\end{block}
\begin{itemize}
\item<3-> Une matrice nulle, $O$, est une matrice dont tous les coefficients sont nuls.
\item<4> \alert{Pourquoi ai-je écrit \textit{\textbf{une}} matrice nulle et non pas \textit{\textbf{la}} matrice nulle?}
\end{itemize}
\end{frame}

\begin{frame}{Soustraction}
\begin{definition}
On définit la soustraction de deux matrices
à partir de l'addition et en utilisant la multiplication par un scalaire comme suit:
\[
\matA-\matB = \matA + (-\matB)
\]
\end{definition}

En vertu des définitions de l'addition de matrices et de multiplication par un scalaire, la
soustraction de matrices peut être faite directement de la façon suivante:
\[
\matA - \matB = \matC \quad {\color{red}\Longleftrightarrow} \quad [a_{ij}] - [b_{ij}] = [a_{ij} - b_{ij}] = [c_{ij}]
\]
\end{frame}

\begin{frame}{Soustraction}
\begin{example}
    Soit les matrices
    \[
    \matA = \begin{pmatrix}
        2 & 3 & 4 \\
        1 & 2 & 1
        \end{pmatrix}
    \qquad \mbox{et} \qquad
    \matB = \begin{pmatrix}
        0 & 2 & 7 \\
        1 & -3 & 5
        \end{pmatrix}
    \]
    Calculez $\matA-3\matB$.
\end{example}
    On vérifiera facilement que la réponse est
    \[
    \matA - 3\matB = \begin{pmatrix}
            2 & -3 & -17 \\
            -2 & 11 & -14
            \end{pmatrix}
    \]
\end{frame}

\begin{frame}{Propriétés}
Soient $\matA, \matB$ et $\matC$ des matrices $m\times n$ et soient $c$ et $d$ des
scalaires.
\begin{itemize}[<+-|alert@+>]
\item $\matA + \matB = \matB + \matA$ \explain{commutativité}
\item $(\matA + \matB) + \matC = \matA + (\matB + \matC) = \matA + \matB + \matC$ \explain{associativité}
\item $\matA + \zero = \zero + \matA = \matA$ \explain{élément neutre}
\item $\matA + (-\matA) = \zero$ \explain{inverse additif}
\item $c(\matA+\matB) = c\matA + c\matB$ \explain{distributivité}
\item $(c+d) \matA = c\matA + d\matA$ \explain{distributivité}
\item $(cd) \matA = c (d\matA)$ \explain{distributivité}
\end{itemize}

\end{frame}


\begin{frame}{Multiplication de deux matrices}
\begin{block}{}<2>
Pour que l'on puisse multiplier une matrice de taille $m\times a$ 
par une matrice
de taille $b\times p$
 il faut que $b=c$ autrement la multiplication n'est pas possible.  
Si la multiplication est possible, on dit que les matrices sont \textbf{compatibles}.

Exemple:\huge
\[
\matA_{m\times \textcolor{red}{n}} \matB_{\textcolor{red}{n}\times p} = \matC_{m\times p}
\]
\end{block}

\end{frame}

\begin{frame}{Multiplication de deux matrices}
Soit
\[
\matA_{1 \times 1} = [2]
\]
\[
\matB_{1\times 1} = \begin{bmatrix}
1
\end{bmatrix}
\]
Calculez $AB$.
\[
\matA_{\textcolor{blue}{1} \times \textcolor{red}{1}}\matB_{\textcolor{red}{1} \times \textcolor{blue}{1}} =  C_{\textcolor{blue}{1} \times \textcolor{blue}{1}}
\]
\[
\matA\matB = [2 \cdot 1] = [2]_{1\times 1}
\]
\end{frame}

\begin{frame}{Multiplication de deux matrices}
Soit
\[
\matA_{1 \times 2} = [2 \quad 5]
\]
\[
\matB_{2\times 1} = \begin{bmatrix}
1 \\ 2
\end{bmatrix}
\]
Calculez $AB$.
\[
\matA_{\textcolor{blue}{1} \times \textcolor{red}{2}}\matB_{\textcolor{red}{2} \times \textcolor{blue}{1}} = C_{\textcolor{blue}{1} \times \textcolor{blue}{1}}
\]
\[
\matA\matB = [2 \cdot 1 + 5\cdot 2] = [12]_{\textcolor{blue}{1}\times \textcolor{blue}{1}}
\]
\end{frame}

\begin{frame}{Multiplication de deux matrices}
Soit
\[
\matA_{1 \times 3} = [2 \quad 5 \quad -3]
\]
\[
\matB_{3\times 1} = \begin{bmatrix}
1 \\ 2 \\ 4
\end{bmatrix}
\]
Calculez $AB$.
\[
\matA_{\textcolor{blue}{1} \times \textcolor{red}{3}}\matB_{\textcolor{red}{3} \times \textcolor{blue}{1}}
= C_{\textcolor{blue}{1} \times \textcolor{blue}{1}}
\]
\[
\matA\matB = [2 \cdot 1 + 5\cdot 2 + (-3)\cdot 4] = [0]_{1\times 1}
\]
\begin{block}{}<2>
\[
\matA\matB = [a_{\textcolor{blue}{1}\textcolor{red}{1}} \cdot b_{\textcolor{red}{1}\textcolor{blue}{1}} 
  + a_{\textcolor{blue}{1}\textcolor{red}{2}}\cdot b_{\textcolor{red}{2}\textcolor{blue}{1}} 
  + a_{\textcolor{blue}{1}\textcolor{red}{3}}\cdot b_{\textcolor{red}{3}\textcolor{blue}{1}}] 
  = [0]_{\textcolor{blue}{1}\times \textcolor{blue}{1}}
\]
\end{block}
\end{frame}

\begin{frame}{Multiplication de deux matrices}
Soit une matrice $\matA_{1\times p}$ et une matrice $\matB_{p\times 1}$.  
Le \textbf{produit} de ces deux matrices, 
$\matA_{1\times\textcolor{red}{p}}\matB_{\textcolor{red}{p}\times 1}$ est une matrice $\matC_{1\times 1}$ dont le seul coefficient est
\huge
\[
[c_{\textcolor{blue}{11}}] = \left[\sum_{\textcolor{red}{k}=1}^{\textcolor{red}{p}} 
a_{\textcolor{blue}{1}\textcolor{red}{k}} 
b_{\textcolor{red}{k}\textcolor{blue}{1}}\right]
\]
\end{frame}


\begin{frame}{Multiplication de deux matrices}
Pour des matrices plus générales, on procède de la même façon, en considérant chaque ligne
de la première matrice multipliée par chaque colonne de la seconde.\vfill

Soit une matrice $\matA_{m\times p}$ et une matrice $\matB_{p\times n}$.  
Le \textbf{produit} de ces deux matrices, 
$\matA_{m\times\textcolor{red}{p}}\matB_{\textcolor{red}{p}\times n}$ est une matrice $\matC_{m\times n}$ telle que
\huge
\[
[c_{\textcolor{blue}{ij}}] = \left[\sum_{\textcolor{red}{k}=1}^{\textcolor{red}{p}} 
a_{\textcolor{blue}{i}\textcolor{red}{k}} 
b_{\textcolor{red}{k}\textcolor{blue}{j}}\right]
\]
\end{frame}

\begin{frame}{Multiplication de deux matrices}
Soit les matrices
$\displaystyle
\matA = \begin{pmatrix}
	1 & 2 & 4 \\
	2 & 6 & 0
	\end{pmatrix}
\qquad
\matB = \begin{pmatrix}
	4 & 1 & 4 & 3\\
	0 & -1 & 3 & 1 \\
	2 & 7 & 5 & 2
	\end{pmatrix}
$;
calculez, si possible, $\matA\matB$ et $\matB\matA$.

\begin{block}{Solution}<2>
Nous avons
\[
\begin{matrix}[rcl]
\matA_{2\times\textcolor{red}{3}}\matB_{\textcolor{red}{3}\times 4} & = & \begin{pmatrix}
	\textcolor{blue}{1} & \textcolor{blue}{2} & \textcolor{blue}{4} \\
	2 & 6 & 0
	\end{pmatrix}
	\begin{pmatrix}
	4 & \textcolor{blue}{1} & 4 & 3\\
	0 & \textcolor{blue}{-1} & 3 & 1 \\
	2 & \textcolor{blue}{7} & 5 & 2
	\end{pmatrix} \\
	\\
	&=& \begin{pmatrix}[cccc]
	4+0+8 & \textcolor{blue}{1-2+28} & 4+6+20 & 3 + 2 + 8 \\
	8+0+0 & 2-6+0 & 8+18+0 & 6+6+0
	\end{pmatrix}\\
	\\
	&=& \begin{pmatrix}
	12 & \textcolor{blue}{27} & 30 & 13 \\
	8 & -4 & 26 & 12
	\end{pmatrix}
	\end{matrix}
\]
\end{block}
\end{frame}

\begin{frame}{Multiplication de deux matrices}
Soit les matrices
$\displaystyle
\matA = \begin{pmatrix}
	1 & 2 & 4 \\
	2 & 6 & 0
	\end{pmatrix}
\qquad
\matB = \begin{pmatrix}
	4 & 1 & 4 & 3\\
	0 & -1 & 3 & 1 \\
	2 & 7 & 5 & 2
	\end{pmatrix}
$;
calculez, si possible, $\matA\matB$ et $\matB\matA$.

\begin{block}{Solution}
$
\matB_{3\times \textcolor{red}{4}}\matA_{\textcolor{red}{2}\times3}$ : les matrices ne sont 
pas compatibles, et le produit n'est pas défini.
\end{block}
\begin{block}{Questions?}<2>
\end{block}
\end{frame}

\begin{frame}{Multiplication de deux matrices}
Même dans les cas où $\matA\matB$ et $\matB\matA$ sont tous les deux définis, la plupart du temps on aura
$\matA\matB \neq \matB\matA$.
\end{frame}

\begin{frame}{Propriétés}
Soit $\matA$ une matrice $m \times n$, et $m, n, p, q$ des entiers arbitraires plus grand ou égal à 1; alors
\begin{itemize}[<+-|alert@+>]
\item $\matA(\matB\matC) = (\matA\matB)\matC$ \explain{où $\matB$ est une matrice $n\times p$ et $\matC$ est une matrice $p\times q$}
\item $\matA(\matB+\matC) = \matA\matB + \matA\matC$ \explain{où $\matB$ et $\matC$ sont des matrices $n\times q$}
\item $(\matB+\matC)\matA$ = $\matB\matA + \matC\matA$ \explain{où $\matB$ et $\matC$ sont des matrices $p\times m$}
\item $k(\matA\matB) = (k\matA)\matB = \matA(k\matB)$ \explain{où $k$ est un scalaire quelconque}
\end{itemize}
\begin{block}{Questions?}<5>
\end{block}
\begin{block}{Démonstration avec calculatrice en ligne}<6>
\end{block}
\end{frame}

\end{document}