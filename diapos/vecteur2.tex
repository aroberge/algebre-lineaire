\documentclass[french, handout]{beamer}
%\documentclass[french]{beamer}

\usepackage{hyperref}

\usetheme{Boadilla}
\usefonttheme{structuresmallcapsserif}

% % %  default French language settings
\usepackage[utf8]{inputenc}			% enable input of accented letters
\usepackage[T1]{fontenc}				% related to above
\usepackage[french]{babel}			% French language support

\usepackage{diapos}

%\usepackage{tikz}					% for drawing figures
%\usetikzlibrary{positioning}
%\tikzset{>=stealth}
%\newcommand{\tikzmark}[3][]{\tikz[overlay,remember picture,baseline] \node [anchor=base,#1](#2) {#3};}

	
\title{Base, image, espace vectoriels}
\author{Math 1253} % shows on bottom of each page by default; students do not care about
    % my name, nor the date in which this was written however, if a revision needs to 
    % be done, having a version number could be useful
\date{version 1}   % shows on bottom of page

\begin{document}
	\frame{\titlepage}
	
	\begin{frame}
	\begin{itemize}
	\item Questions?
	\item Revue de la méthode de Gauss-Jordan
	\item Revue: ensembles, éléments, sous-ensembles
	\item Image d'une matrice
	\item Noyau d'une matrice
	\item Espace vectoriel
	\item Sous-espace vectoriel
	\end{itemize}
\end{frame}	


\begin{frame}{Revue: méthode de Gauss-Jordan}

Important: je ne montre ici que la \textbf{matrice des coefficients.}
\vfill

Au besoin, on fait un échange de ligne pour s'assurer
que $p_1$ est différent de zéro.

\[
\begin{pmatrix}
p_1 & *  & * & * & * & * & * & * & * \\
* & *  & * & * & * & * & * & * & * \\
* & *  & * & * & * & * & * & * & * \\
* & *  & * & * & * & * & * & * & * \\
* & *  & * & * & * & * & * & * & * \\
* & *  & * & * & * & * & * & * & *
\end{pmatrix}
\]

\end{frame}	
	
\begin{frame}{Revue: méthode de Gauss-Jordan}

On fait des opérations sur les lignes de la forme 
\Huge
\[
 L_j + a L_{\color{red}1} \rightarrow L_j
\]
\normalsize
pour avoir des zéros dans la première colonne sous le premier pivot.

\[
\begin{pmatrix}
p_1 & *  & * & * & * & * & * & * & * \\
{\color{red}{0}} & * & * & * & * & * & * & * & * \\
{\color{red}{0}} & * & * & * & * & * & * & * & * \\
{\color{red}{0}} & * & * & * & * & * & * & * & * \\
{\color{red}{0}} & * & * & * & * & * & * & * & * \\
{\color{red}{0}} & * & * & * & * & * & * & * & * 
\end{pmatrix}
\]
\end{frame}	

\begin{frame}{Revue: méthode de Gauss-Jordan}

On identifie une ligne avec le deuxième pivot; ici,
il y a deux candidats possibles ($a$, $b$ et $c$ sont
différents de zéro)

\[
\begin{pmatrix}
p_1 & *  & * & * & * & * & * & * & * \\
{\color{red}{0}} & {\color{red}{0}} & {\color{red}{0}} & * & * & * & * & * & * \\
{\color{red}{0}} & {\color{red}{0}} & a & * & * & * & * & * & * \\
{\color{red}{0}} & {\color{red}{0}} & b & * & * & * & * & * & * \\
{\color{red}{0}} & {\color{red}{0}} & {\color{red}{0}} & * & * & * & * & * & * \\
{\color{red}{0}} & {\color{red}{0}} & c & * & * & * & * & * & * 
\end{pmatrix}
\]
Ici on peut faire l'échange suivant:
\Huge
\[
 L_3 \leftrightarrow L_2
\]
\normalsize
et on va écrire $p_2$ au lieu de $a$
\end{frame}	
	
\begin{frame}{Revue: méthode de Gauss-Jordan}
\[
\begin{pmatrix}
p_1 & *  & * & * & * & * & * & * & * \\
{\color{red}{0}} & {\color{red}{0}} & p_2 & * & * & * & * & * & * \\
{\color{red}{0}} & {\color{red}{0}} & {\color{red}{0}} & * & * & * & * & * & * \\

{\color{red}{0}} & {\color{red}{0}} & b & * & * & * & * & * & * \\
{\color{red}{0}} & {\color{red}{0}} & {\color{red}{0}} & * & * & * & * & * & * \\
{\color{red}{0}} & {\color{red}{0}} & c & * & * & * & * & * & * 
\end{pmatrix}
\]
\end{frame}		
	
\begin{frame}{Revue: méthode de Gauss-Jordan}
On fait des opérations sur les lignes de la forme 
\Huge
\[
 L_j + a L_{\color{red}2} \rightarrow L_j \qquad(j > 2)
\]
\normalsize
pour avoir des zéros dans la colonne sous le deuxième pivot.
\[
\begin{pmatrix}
p_1 & *  & * & * & * & * & * & * & * \\
{\color{red}{0}} & {\color{red}{0}} & p_2 & * & * & * & * & * & * \\
{\color{red}{0}} & {\color{red}{0}} & {\color{red}{0}} & * & * & * & * & * & * \\

{\color{red}{0}} & {\color{red}{0}} & {\color{red}{0}}  & * & * & * & * & * & * \\
{\color{red}{0}} & {\color{red}{0}} & {\color{red}{0}} & * & * & * & * & * & * \\
{\color{red}{0}} & {\color{red}{0}} & {\color{red}{0}}  & * & * & * & * & * & * 
\end{pmatrix}
\]
\end{frame}		
	
\begin{frame}{Revue: méthode de Gauss-Jordan}
On continue ainsi en faisant uniquement les opérations suivantes: 
(Ligne de pivot: $p$)
\Huge
\[
\begin{matrix}[lr]
 L_j + a L_{\color{red}p} \rightarrow L_j &\qquad(j > p)\\[25pt]
 L_i \leftrightarrow L_j &\qquad(i, j >= p)\\
 a L_i \rightarrow L_i &\qquad(i >= p)
\end{matrix}
\]
\normalsize
\end{frame}			
	
	
\begin{frame}{Revue: méthode de Gauss-Jordan}
On continue jusqu'à avoir une forme échelonnée	
\[
\begin{pmatrix}
p_1 & *  & * & * & * & * & * & * & * \\
{\color{red}{0}} & {\color{red}{0}} & p_2 & * & * & * & * & * & * \\
{\color{red}{0}} & {\color{red}{0}} & {\color{red}{0}} & p_3 & * & * & * & * & * \\ 
{\color{red}{0}} & {\color{red}{0}} & {\color{red}{0}} & {\color{red}{0}} & {\color{red}{0}} & {\color{red}{0}} & p_4 & * & * \\ 
{\color{red}{0}} & {\color{red}{0}} & {\color{red}{0}} & {\color{red}{0}} & {\color{red}{0}} & {\color{red}{0}} & {\color{red}{0}} & {\color{red}{0}} & p_5 \\ 
{\color{red}{0}} & {\color{red}{0}} & {\color{red}{0}} & {\color{red}{0}} & {\color{red}{0}} & {\color{red}{0}} & {\color{red}{0}} & {\color{red}{0}} & {\color{red}{0}} 
\end{pmatrix}
\]	
\end{frame}	

\begin{frame}{Revue: méthode de Gauss-Jordan}
Si la matrice des coefficients a une, ou plusieurs lignes nulles	
\[
\begin{pmatrix}[rrrrrrrrr|r]
p_1 & *  & * & * & * & * & * & * & * &* \\
{\color{red}{0}} & {\color{red}{0}} & p_2 & * & * & * & * & * & * & * \\
{\color{red}{0}} & {\color{red}{0}} & {\color{red}{0}} & p_3 & * & * & * & * & * & *\\ 
{\color{red}{0}} & {\color{red}{0}} & {\color{red}{0}} & {\color{red}{0}} & {\color{red}{0}} & {\color{red}{0}} & p_4 & * & * & * \\ 
{\color{red}{0}} & {\color{red}{0}} & {\color{red}{0}} & {\color{red}{0}} & {\color{red}{0}} & {\color{red}{0}} & {\color{red}{0}} & {\color{red}{0}} & p_5 & * \\ 
{\color{red}{0}} & {\color{red}{0}} & {\color{red}{0}} & {\color{red}{0}} & {\color{red}{0}} & {\color{red}{0}} & {\color{red}{0}} & {\color{red}{0}} & {\color{red}{0}} & b
\end{pmatrix}
\]	
Chaque ligne nulle correspond à une équation de la forme $0=b$. Si $b\neq 0$, une telle équation est fausse et on a un système incompatible. Il est inutile de poursuivre les calculs
\vfill
{\color{red}Super important:} Avant de conclure quoi que ce soit, \textbf{toujours} exprimer l'information sous forme d'équation \textbf{avec} les variables (inconnues).
\end{frame}	



\begin{frame}{Revue: méthode de Gauss-Jordan}
Par la suite, on procède de gauche à droite avec des transformations comme\footnote{On peut attendre à plus 
tard avant de s'assurer que chaque pivot soit égal à 1.}
\Large
\[
\frac{1}{p_i} L_i \rightarrow L_i
\]
\normalsize
À partir d'ici, on peut ignorer (omettre) toute ligne nulle	
\[
\begin{pmatrix}
1 & *  & * & * & * & * & * & * & * \\
{\color{red}{0}} & {\color{red}{0}} & 1 & * & * & * & * & * & * \\
{\color{red}{0}} & {\color{red}{0}} & {\color{red}{0}} & 1 & * & * & * & * & * \\ 
{\color{red}{0}} & {\color{red}{0}} & {\color{red}{0}} & {\color{red}{0}} & {\color{red}{0}} & {\color{red}{0}} & 1 & * & * \\ 
{\color{red}{0}} & {\color{red}{0}} & {\color{red}{0}} & {\color{red}{0}} & {\color{red}{0}} & {\color{red}{0}} & {\color{red}{0}} & {\color{red}{0}} & 1 \\ 
\end{pmatrix}
\]	
\end{frame}	

\begin{frame}{Revue: méthode de Gauss-Jordan}
Pour chaque ligne avec un pivot, $p_i$, en commençant avec le pivot
le plus à droite ($i=5$ ici), on fera
\Large
\[
 L_j + a L_{\color{red}i} \rightarrow L_j \qquad(j <i)
\]
\normalsize
	
\[
\begin{pmatrix}
1 & *  & * & * & * & * & * & * & {\color{blue}{0}} \\
{\color{red}{0}} & {\color{red}{0}} & 1 & * & * & * & * & * & {\color{blue}{0}} \\
{\color{red}{0}} & {\color{red}{0}} & {\color{red}{0}} & 1 & * & * & * & * & {\color{blue}{0}} \\ 
{\color{red}{0}} & {\color{red}{0}} & {\color{red}{0}} & {\color{red}{0}} & {\color{red}{0}} & {\color{red}{0}} & 1 & * & {\color{blue}{0}} \\ 
{\color{red}{0}} & {\color{red}{0}} & {\color{red}{0}} & {\color{red}{0}} & {\color{red}{0}} & {\color{red}{0}} & {\color{red}{0}} & {\color{red}{0}} & 1
\end{pmatrix}
\]	
\end{frame}	
	
	
\begin{frame}{Revue: méthode de Gauss-Jordan}
On continue jusqu'à avoir obtenu la forme échelonnée réduite
	
\[
\begin{pmatrix}
1 & * & {\color{blue}{0}} & {\color{blue}{0}} & * & * & {\color{blue}{0}} & * & {\color{blue}{0}}  \\
{\color{red}{0}} & {\color{red}{0}} & 1 & {\color{blue}{0}} & * & * & {\color{blue}{0}} & * & {\color{blue}{0}}  \\
{\color{red}{0}} & {\color{red}{0}} & {\color{red}{0}} & 1 & * & * & {\color{blue}{0}} & * & {\color{blue}{0}}  \\ 
{\color{red}{0}} & {\color{red}{0}} & {\color{red}{0}} & {\color{red}{0}} & {\color{red}{0}} & {\color{red}{0}} & 1 & * & {\color{blue}{0}}  \\ 
{\color{red}{0}} & {\color{red}{0}} & {\color{red}{0}} & {\color{red}{0}} & {\color{red}{0}} & {\color{red}{0}} & {\color{red}{0}} & {\color{red}{0}} & 1
\end{pmatrix}
\]	
\vfill

On identifie les variables libres ($x_2, x_5, x_6, x_8$) pouvant prendre n'importe quelle valeur, et les variables dépendantes ($x_1, x_3, x_4, x_7, x_9$).
\end{frame}	
		
\begin{frame}{Méthode de Gauss-Jordan: une suggestion}

Supposons que j'ai
\[
\begin{bmatrix}[rrr|r]
2 & 3 & -4 & 6\\
1 & -2 & 3 & 5
\end{bmatrix}
\]

On veut faire $L_2 -2L_1 \rightarrow L_2$. Écrivez "en marge"
\[
\begin{matrix}[rrrr|r]
{\color{red}-2L_1:} & -4 & -6 & 8 & -12\\
{\color{red}L_2:} & 1 & -2 & 3 & 5 \\
\hline
& -3 & -8 & 11 & -7
\end{matrix}
\]

\vfill
N'oubliez pas d'inclure \textbf{tous} les coefficients, y compris ceux
à droite de la barre verticale.

\end{frame}		
	
	
\begin{frame}{Revue}

\begin{itemize}
\item Ensembles
	\begin{itemize}
	\item $\mathbb{E} = \{ 1, 3, 5, 7, 9\}$
	\item $\mathbb{F} = \{3, 9\}$
	\item $\mathbb{G} = \{2, 3, 4\}$
	\end{itemize}
\item Appartenance
	\begin{itemize}
	\item Un élément \textbf{appartient} ou n'appartient pas à un ensemble
		\begin{itemize}
		\item $5 \in \mathbb{E}$
		\item $5 \notin \mathbb{G}$
		\end{itemize}
	\item Un ensemble peut être un \textbf{sous-ensemble} d'un autre ensemble.
		\begin{itemize}
		\item $\mathbb{F} \subset \mathbb{E}$
		\item $\mathbb{G} \not \subset \mathbb{E}$
		\item $\mathbb{E} \subset \mathbb{E}$		
		\end{itemize}
	\end{itemize}
\end{itemize}

\end{frame}

\begin{frame}{Revue}

\begin{itemize}
\item  $\mathbb{V} = \{\mat{a}_1, \ldots, \mat{a}_n\}$ est l'ensemble des
vecteurs $\mat{a}_1$, $\mat{a}_2$, \ldots, $\mat{a}_n$
	\begin{itemize}
	\item  $\mathbb{V}$ a $n$ éléments
	\end{itemize}
\item $\mathbb{W} = \operatorname{Vect}\{\mat{a}_1, \ldots, \mat{a}_n\}$ est l'ensemble de toutes les \textbf{combinaisons linéaires} des
vecteurs $\mat{a}_1$ à $\mat{a}_n$
	\begin{itemize}
	\item Combinaison linéaire: $x_1\mat{a}_1 + \ldots + x_n\mat{a}_n$
	\item $x_1$ peut prendre n'importe quelle valeur; c'est la même chose pour tous les $x_i$.
		\begin{itemize}
		\item Il y a un nombre infini de combinaisons linéaires possibles
		\item $\mathbb{W}$ a un nombre infini d'éléments
		\end{itemize}
	\end{itemize}
\item $\mathbb{V} \subset \mathbb{W}$
\end{itemize}

\end{frame}		
	
	
\begin{frame}{Image}
L'\textbf{image} d'une matrice $\mat{A}$, dénoté par $\operatorname{Im}\mat{A}$, est l'ensemble de toutes les combinaisons linéaires
des colonnes de $\mat{A}$.

Exemple:
\[
\mat{A} = \begin{bmatrix}
1 & 4 & 7\\
2 & 5 & 8\\
3 & 6 & 9
\end{bmatrix}
=
\begin{bmatrix}
\mat{a}_1 & \mat{a}_2 & \mat{a}_3
\end{bmatrix}
\]

\[
\mat{a}_1 = \begin{bmatrix}
1 \\ 2 \\ 3
\end{bmatrix}
\qquad
\mat{a}_2 = \begin{bmatrix}
4 \\ 5\\6
\end{bmatrix}
\qquad
\mat{a}_3 = \begin{bmatrix}
7\\8\\9
\end{bmatrix}
\]
\[
\operatorname{Im}\mat{A} = \operatorname{Vect}\{\mat{a}_1, \mat{a}_2, \mat{a}_3\}
\]
\end{frame}	


\begin{frame}{Image}

Question typique: Est-ce que le vecteur $\mat{b}$ appartient à l'image
de $\mat{A}$?
\vfill

Ceci revient à poser la question suivante: Est-ce qu'on peut écrire
$\mat{b}$ comme une combinaison linéaire des vecteurs $\mat{a}_1, \mat{a}_2, \mat{a}_3$?

\vfill
Ceci est équivalent à la question suivante: Existe-t-il des valeurs de
$x_1, x_2, x_3$ telle que l'on peut écrire
\[
\mat{b} = x_1\mat{a}_1 + x_2 \mat{a}_2 + x_3 \mat{a}_3
\]
\end{frame}		

\begin{frame}{Noyau}

On appelle \textbf{noyau}\footnote{En anglais, on dit le \textit{kernel}.} d'une matrice $\mat{A}$ l'ensemble $\operatorname{Ker}\mat{A}$ des solutions
de l'équation homogène
\[
\mat{A}\mat{x} = \mat{0}
\]

\vfill
Question typique: {\color{blue} Trouvez le noyau de $\mat{A}$}
ou encore {\color{blue} Trouvez $\operatorname{Ker}\mat{A}$} veut dire la
même chose que \textit{Trouvez toutes les solutions de l'équation} 
$\mat{A}\mat{x} = \mat{0}$.
\end{frame}		
	
\begin{frame}{Espace vectoriel}
Un \textbf{espace vectoriel} est un ensemble $V$ d'objets appelés \textit{vecteurs}, sur lesquels on définit
deux opérations, soit \textit{l'addition} ainsi que \textit{la multiplication par un scalaire}, et pour lequel
les axiomes suivant sont satisfaits pour tous les vecteurs $\mat{u}, \mat{v}, \mat{w}$ dans $V$
et pour tous les scalaires $\alpha, \beta$.
\begin{enumerate}
\item {\color{red} $\mat{u}+\mat{v}\in V$}
\item  $\mat{u} + \mat{v} = \mat{v} + \mat{u}$
\item  $ (\mat{u} + \mat{v}) + \mat{w} = \mat{u} + (\mat{v} + \mat{w})$
\item {\color{red} $\exists\zero\in V:$} $ \mat{u} + \zero = \mat{u}$
\item $\forall \mat{u}\in V\quad \exists-\mat{u}\in V: \mat{u} + (-\mat{u}) = \zero$.
\item {\color{red} $\alpha\mat{u} \in V$}
\item $\alpha(\mat{u} + \mat{v}) = \alpha\mat{u} + \alpha\mat{v}$
\item  $ (\alpha + \beta)\mat{u} = \alpha\mat{u} + \beta\mat{u}$
\item  $\alpha(\beta\mat{u}) = (\alpha\beta)\mat{u}$
\item  $1\mat{u} = \mat{u}$
\end{enumerate}
\end{frame}

\begin{frame}{Espace vectoriel}
\begin{itemize}
\item $\BBR^n$ est un espace vectoriel
\item $\displaystyle
\operatorname{Vect}\left\{
\begin{bmatrix}
1 \\ 0
\end{bmatrix}
,\begin{bmatrix}
0 \\ 1
\end{bmatrix}
\right\}
$
est un espace vectoriel (en fait, ceci est $\BBR^2$)
\end{itemize}
\end{frame}

\begin{frame}{Sous-espace vectoriel}
Soit $W$ un \textit{sous-ensemble} d'un espace vectoriel $V$.  
On appellera $W$ un \textbf{sous-espace vectoriel}\footnote{On omet parfois le mot \textit{vectoriel}
pour simplement écrire \textbf{sous-espace} pour signifier la même chose.} de $V$ si
les trois propriétés suivantes sont satisfaites:
\begin{enumerate}
\item Le vecteur nul de $V$ est dans $W$.
\item $W$ est fermé pour l'addition; $\mat{u}, \mat{w} \in W \Rightarrow \mat{u} + \mat{w} \in W$.
\item $W$ est fermé pour la multiplication par un scalaire: $\mat{w}\in W \Rightarrow a\mat{w}\in W$ pour $a\in \BBR$,
l'ensemble de scalaires utilisés pour la définition de $V$.
\end{enumerate}

On peut facilement vérifier qu'un \textbf{sous-ensemble} $W$ de $V$ qui satisfait les trois propriétés mentionnées dans la définition ci-dessus est un espace vectoriel.
\end{frame}

\begin{frame}{Sous-espace vectoriel}
Note: on peut vérifier les propriétés trois dans une seule étape:
\[
\left.
\begin{matrix}
\mat{u}, \mat{w} \in W \\
a, b \in \BBR
\end{matrix}
\right\}
\Rightarrow a\mat{u} + b\mat{w} \in W
\]
\end{frame}

\begin{frame}{Sous-espace vectoriel}
Exemple: l'ensemble de tous les points dans le plan $x-y$ est
un sous-espace de $\BBR^3$

\[
\operatorname{Vect}\left\{
\begin{bmatrix}
1\\0\\0
\end{bmatrix},
\begin{bmatrix}
0\\1\\0
\end{bmatrix}
\right\}
\]
\end{frame}

\begin{frame}{Sous-espace vectoriel}
Exemple: $\BBR^2$ est un espace vectoriel mais ce n'est pas 
un sous-espace vectoriel de $\BBR^3$

\[
\begin{bmatrix}
x \\ y
\end{bmatrix}
\notin \BBR^3
\]
\end{frame}

\begin{frame}{Sous-espace vectoriel}
L'ensemble des vecteurs de la forme
$\displaystyle
\begin{bmatrix}[c]
t\\
t+1
\end{bmatrix}
$ correspondant à la droite $y=x+1$ ne forme pas un sous-espace de $\BBR^2$.
		\begin{tikzpicture}
			% les axes
			\draw[->] (-1,0) -- (4,0) node[anchor=west]{\color{gray}x};
			\draw[->] (0,-1) -- (0,5) node[anchor=south]{\color{gray}y};
			% la droite
			\draw[-, thick, blue] (0,-1) -- (5,4) node[anchor=east]{\color{blue}\small $y=x+1$};
		\end{tikzpicture}
\end{frame}

\begin{frame}{Sous-espace vectoriel}
L'ensemble des vecteurs de la forme
$\displaystyle
\begin{bmatrix}
t\\
2t
\end{bmatrix}
$ correspondant à la droite $y=2x$ forme un sous-espace de $\BBR^2$.
		\begin{tikzpicture}
			% les axes
			\draw[->] (-2,0) -- (3,0) node[anchor=west]{\color{gray}x};
			\draw[->] (0,-2) -- (0,4) node[anchor=south]{\color{gray}y};
			% la droite
			\draw[-, thick, blue] (-1,-2) -- (2,4) node[anchor=east]{\color{blue}\small $y=2x$};
		\end{tikzpicture}
\end{frame}

\end{document}
