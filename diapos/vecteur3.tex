\documentclass[french, handout]{beamer}
%\documentclass[french]{beamer}

\usepackage{hyperref}

\usetheme{Boadilla}
\usefonttheme{structuresmallcapsserif}

% % %  default French language settings
\usepackage[utf8]{inputenc}			% enable input of accented letters
\usepackage[T1]{fontenc}				% related to above
\usepackage[french]{babel}			% French language support

\usepackage{diapos}


	
\title{Base, image, espace vectoriels}
\author{Math 1253} % shows on bottom of each page by default; students do not care about
    % my name, nor the date in which this was written however, if a revision needs to 
    % be done, having a version number could be useful
\date{version 1}   % shows on bottom of page

\begin{document}

% Quick addendum to hand-written class notes.

\begin{frame}{Exemple 1}
\begin{example}
Soit le système d'équations linéaires
\[
\left\{ \begin{matrix}
x &=& 1\\
y &=& 2
\end{matrix}\right.
\]
Quelle est la dimension de l'espace des solutions?  S'agit-il d'un espace vectoriel?
\end{example}

La solution unique de ce système est un point dans $\BBR^2$;
la dimension de cet espace est donc 0.  Ce point peut être représenté par un vecteur unique
qui n'est pas le vecteur nul; par conséquent, ce n'est pas un espace vectoriel.

\end{frame}

\begin{frame}{Exemple 2}

\begin{example}
Soit l'équation linéaire
\[
x + y = b
\]
Quelle est la dimension de l'espace des solutions?  S'agit-il d'un espace vectoriel?
\end{example}


Ceci est l'équation d'une droite; l'espace des solutions est donc de dimension 1.

On peut écrire la matrice augmentée ainsi:
\[
\begin{bmatrix}[rr|r]
\textcircled{1} & 1 & b\\
\end{bmatrix}
\]
On a une variable libre ($y$) qu'on peut paramétrer par $t$.

On a donc $x = -t + b$.

Sous forme paramétrique vectorielle, la solution est

\[
\begin{pmatrix}
x \\ y
\end{pmatrix}
= \begin{pmatrix}
b-t \\
t
\end{pmatrix}
=
\begin{pmatrix}
b\\0
\end{pmatrix}
+ t\begin{pmatrix}
-1 \\ 1
\end{pmatrix}
\]

\end{frame}

\begin{frame}{Exemple 2 - suite}
On a
\[
\begin{pmatrix}
x \\ y
\end{pmatrix}
= 
\begin{pmatrix}
b\\0
\end{pmatrix}
+ t\begin{pmatrix}
-1 \\ 1
\end{pmatrix}
\]

Pour que ceci soit un espace vectoriel, il faut que le vecteur nul appartienne à l'espace des solutions, 
ce qui est possible seulement si $b=0$.  Dans ce cas, l'espace des solutions sera engendré par le vecteur
$\displaystyle
\begin{pmatrix}
-1 \\ 1
\end{pmatrix}
$

\begin{block}{Important}
L'espace des \textbf{solutions} de $\mat{A}\mat{x} = \mat{b}$ n'est
pas un espace vectoriel, sauf si $\mat{b}=\mat{0}$, auquel
cas cet espace des solutions est dénoté par
 $\operatorname{Ker}\mat{A}$.
\end{block}

\end{frame}


\end{document}
